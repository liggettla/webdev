\documentclass[]{book}
\usepackage{lmodern}
\usepackage{amssymb,amsmath}
\usepackage{ifxetex,ifluatex}
\usepackage{fixltx2e} % provides \textsubscript
\ifnum 0\ifxetex 1\fi\ifluatex 1\fi=0 % if pdftex
  \usepackage[T1]{fontenc}
  \usepackage[utf8]{inputenc}
\else % if luatex or xelatex
  \ifxetex
    \usepackage{mathspec}
  \else
    \usepackage{fontspec}
  \fi
  \defaultfontfeatures{Ligatures=TeX,Scale=MatchLowercase}
\fi
% use upquote if available, for straight quotes in verbatim environments
\IfFileExists{upquote.sty}{\usepackage{upquote}}{}
% use microtype if available
\IfFileExists{microtype.sty}{%
\usepackage{microtype}
\UseMicrotypeSet[protrusion]{basicmath} % disable protrusion for tt fonts
}{}
\usepackage[margin=1in]{geometry}
\usepackage{hyperref}
\hypersetup{unicode=true,
            pdftitle={Bootstrap and React for Web Development},
            pdfauthor={L A Liggett},
            pdfborder={0 0 0},
            breaklinks=true}
\urlstyle{same}  % don't use monospace font for urls
\usepackage{natbib}
\bibliographystyle{apalike}
\usepackage{color}
\usepackage{fancyvrb}
\newcommand{\VerbBar}{|}
\newcommand{\VERB}{\Verb[commandchars=\\\{\}]}
\DefineVerbatimEnvironment{Highlighting}{Verbatim}{commandchars=\\\{\}}
% Add ',fontsize=\small' for more characters per line
\usepackage{framed}
\definecolor{shadecolor}{RGB}{248,248,248}
\newenvironment{Shaded}{\begin{snugshade}}{\end{snugshade}}
\newcommand{\KeywordTok}[1]{\textcolor[rgb]{0.13,0.29,0.53}{\textbf{#1}}}
\newcommand{\DataTypeTok}[1]{\textcolor[rgb]{0.13,0.29,0.53}{#1}}
\newcommand{\DecValTok}[1]{\textcolor[rgb]{0.00,0.00,0.81}{#1}}
\newcommand{\BaseNTok}[1]{\textcolor[rgb]{0.00,0.00,0.81}{#1}}
\newcommand{\FloatTok}[1]{\textcolor[rgb]{0.00,0.00,0.81}{#1}}
\newcommand{\ConstantTok}[1]{\textcolor[rgb]{0.00,0.00,0.00}{#1}}
\newcommand{\CharTok}[1]{\textcolor[rgb]{0.31,0.60,0.02}{#1}}
\newcommand{\SpecialCharTok}[1]{\textcolor[rgb]{0.00,0.00,0.00}{#1}}
\newcommand{\StringTok}[1]{\textcolor[rgb]{0.31,0.60,0.02}{#1}}
\newcommand{\VerbatimStringTok}[1]{\textcolor[rgb]{0.31,0.60,0.02}{#1}}
\newcommand{\SpecialStringTok}[1]{\textcolor[rgb]{0.31,0.60,0.02}{#1}}
\newcommand{\ImportTok}[1]{#1}
\newcommand{\CommentTok}[1]{\textcolor[rgb]{0.56,0.35,0.01}{\textit{#1}}}
\newcommand{\DocumentationTok}[1]{\textcolor[rgb]{0.56,0.35,0.01}{\textbf{\textit{#1}}}}
\newcommand{\AnnotationTok}[1]{\textcolor[rgb]{0.56,0.35,0.01}{\textbf{\textit{#1}}}}
\newcommand{\CommentVarTok}[1]{\textcolor[rgb]{0.56,0.35,0.01}{\textbf{\textit{#1}}}}
\newcommand{\OtherTok}[1]{\textcolor[rgb]{0.56,0.35,0.01}{#1}}
\newcommand{\FunctionTok}[1]{\textcolor[rgb]{0.00,0.00,0.00}{#1}}
\newcommand{\VariableTok}[1]{\textcolor[rgb]{0.00,0.00,0.00}{#1}}
\newcommand{\ControlFlowTok}[1]{\textcolor[rgb]{0.13,0.29,0.53}{\textbf{#1}}}
\newcommand{\OperatorTok}[1]{\textcolor[rgb]{0.81,0.36,0.00}{\textbf{#1}}}
\newcommand{\BuiltInTok}[1]{#1}
\newcommand{\ExtensionTok}[1]{#1}
\newcommand{\PreprocessorTok}[1]{\textcolor[rgb]{0.56,0.35,0.01}{\textit{#1}}}
\newcommand{\AttributeTok}[1]{\textcolor[rgb]{0.77,0.63,0.00}{#1}}
\newcommand{\RegionMarkerTok}[1]{#1}
\newcommand{\InformationTok}[1]{\textcolor[rgb]{0.56,0.35,0.01}{\textbf{\textit{#1}}}}
\newcommand{\WarningTok}[1]{\textcolor[rgb]{0.56,0.35,0.01}{\textbf{\textit{#1}}}}
\newcommand{\AlertTok}[1]{\textcolor[rgb]{0.94,0.16,0.16}{#1}}
\newcommand{\ErrorTok}[1]{\textcolor[rgb]{0.64,0.00,0.00}{\textbf{#1}}}
\newcommand{\NormalTok}[1]{#1}
\usepackage{longtable,booktabs}
\usepackage{graphicx,grffile}
\makeatletter
\def\maxwidth{\ifdim\Gin@nat@width>\linewidth\linewidth\else\Gin@nat@width\fi}
\def\maxheight{\ifdim\Gin@nat@height>\textheight\textheight\else\Gin@nat@height\fi}
\makeatother
% Scale images if necessary, so that they will not overflow the page
% margins by default, and it is still possible to overwrite the defaults
% using explicit options in \includegraphics[width, height, ...]{}
\setkeys{Gin}{width=\maxwidth,height=\maxheight,keepaspectratio}
\IfFileExists{parskip.sty}{%
\usepackage{parskip}
}{% else
\setlength{\parindent}{0pt}
\setlength{\parskip}{6pt plus 2pt minus 1pt}
}
\setlength{\emergencystretch}{3em}  % prevent overfull lines
\providecommand{\tightlist}{%
  \setlength{\itemsep}{0pt}\setlength{\parskip}{0pt}}
\setcounter{secnumdepth}{5}
% Redefines (sub)paragraphs to behave more like sections
\ifx\paragraph\undefined\else
\let\oldparagraph\paragraph
\renewcommand{\paragraph}[1]{\oldparagraph{#1}\mbox{}}
\fi
\ifx\subparagraph\undefined\else
\let\oldsubparagraph\subparagraph
\renewcommand{\subparagraph}[1]{\oldsubparagraph{#1}\mbox{}}
\fi

%%% Use protect on footnotes to avoid problems with footnotes in titles
\let\rmarkdownfootnote\footnote%
\def\footnote{\protect\rmarkdownfootnote}

%%% Change title format to be more compact
\usepackage{titling}

% Create subtitle command for use in maketitle
\providecommand{\subtitle}[1]{
  \posttitle{
    \begin{center}\large#1\end{center}
    }
}

\setlength{\droptitle}{-2em}

  \title{Bootstrap and React for Web Development}
    \pretitle{\vspace{\droptitle}\centering\huge}
  \posttitle{\par}
    \author{L A Liggett}
    \preauthor{\centering\large\emph}
  \postauthor{\par}
      \predate{\centering\large\emph}
  \postdate{\par}
    \date{2019-08-15}

\usepackage{booktabs}
\usepackage{amsthm}
\makeatletter
\def\thm@space@setup{%
  \thm@preskip=8pt plus 2pt minus 4pt
  \thm@postskip=\thm@preskip
}
\makeatother

\begin{document}
\maketitle

{
\setcounter{tocdepth}{1}
\tableofcontents
}
\chapter{Introduction}\label{introduction}

\chapter{HTML}\label{html}

\section{HTML Properties}\label{html-properties}

Commenting in HTML.

\begin{Shaded}
\begin{Highlighting}[]
\OperatorTok{<}\NormalTok{!}\ExtensionTok{--}
\ExtensionTok{These}\NormalTok{ are some comments.}
\ExtensionTok{--}\OperatorTok{>}
\end{Highlighting}
\end{Shaded}

The head tag allows metadata to be labeled, the text of title for
instance is typically listed in the tab or the status bar of the page in
a browser.

\begin{Shaded}
\begin{Highlighting}[]
\OperatorTok{<}\NormalTok{!}\ExtensionTok{DOCTYPE}\NormalTok{ html}\OperatorTok{>}         
\OperatorTok{<}\ExtensionTok{html}\OperatorTok{>}                  
    \OperatorTok{<}\FunctionTok{head}\OperatorTok{>}              
        \OperatorTok{<}\ExtensionTok{title}\OperatorTok{>}         
            \ExtensionTok{My}\NormalTok{ Web Page!}
        \OperatorTok{<}\NormalTok{/}\ExtensionTok{title}\OperatorTok{>}        
    \OperatorTok{<}\NormalTok{/}\ExtensionTok{head}\OperatorTok{>}             
\OperatorTok{<}\NormalTok{/}\ExtensionTok{html}\OperatorTok{>}                 
\end{Highlighting}
\end{Shaded}

The body specifies text for the page body.

\begin{Shaded}
\begin{Highlighting}[]
\OperatorTok{<}\NormalTok{!}\ExtensionTok{DOCTYPE}\NormalTok{ html}\OperatorTok{>}         
\OperatorTok{<}\ExtensionTok{html}\OperatorTok{>}                  
    \OperatorTok{<}\ExtensionTok{body}\OperatorTok{>}              
        \ExtensionTok{Hello}\NormalTok{, world!   }
    \OperatorTok{<}\NormalTok{/}\ExtensionTok{body}\OperatorTok{>}             
\OperatorTok{<}\NormalTok{/}\ExtensionTok{html}\OperatorTok{>}                 
\end{Highlighting}
\end{Shaded}

Headings specifies header text of increasingly small sizes.

\begin{Shaded}
\begin{Highlighting}[]
\OperatorTok{<}\NormalTok{!}\ExtensionTok{DOCTYPE}\NormalTok{ html}\OperatorTok{>}         
\OperatorTok{<}\ExtensionTok{html}\OperatorTok{>}                  
    \OperatorTok{<}\ExtensionTok{body}\OperatorTok{>}              
        \OperatorTok{<}\ExtensionTok{h1}\OperatorTok{>}\NormalTok{This is the largest headline}\OperatorTok{<}\NormalTok{/h1}\OperatorTok{>}        
        \OperatorTok{<}\ExtensionTok{h2}\OperatorTok{>}\NormalTok{This is also a large headline}\OperatorTok{<}\NormalTok{/h2}\OperatorTok{>}       
        \OperatorTok{<}\ExtensionTok{h3}\OperatorTok{>}\NormalTok{This is a slightly smaller headline}\OperatorTok{<}\NormalTok{/h3}\OperatorTok{>} 
        \OperatorTok{<}\ExtensionTok{h4}\OperatorTok{>}\NormalTok{This is an even smaller headline}\OperatorTok{<}\NormalTok{/h4}\OperatorTok{>}    
        \OperatorTok{<}\ExtensionTok{h5}\OperatorTok{>}\NormalTok{This is the second-smallest headline}\OperatorTok{<}\NormalTok{/h5}\OperatorTok{>}
        \OperatorTok{<}\ExtensionTok{h6}\OperatorTok{>}\NormalTok{This is the smallest headline}\OperatorTok{<}\NormalTok{/h6}\OperatorTok{>}       
    \OperatorTok{<}\NormalTok{/}\ExtensionTok{body}\OperatorTok{>}             
\OperatorTok{<}\NormalTok{/}\ExtensionTok{html}\OperatorTok{>}                 
\end{Highlighting}
\end{Shaded}

Unordered lists specify bullet points.

\begin{Shaded}
\begin{Highlighting}[]
\OperatorTok{<}\NormalTok{!}\ExtensionTok{DOCTYPE}\NormalTok{ html}\OperatorTok{>}         
\OperatorTok{<}\ExtensionTok{html}\OperatorTok{>}                  
    \OperatorTok{<}\ExtensionTok{body}\OperatorTok{>}              
        \ExtensionTok{An}\NormalTok{ Unordered List:           }
        \OperatorTok{<}\ExtensionTok{ul}\OperatorTok{>}                         
            \OperatorTok{<}\ExtensionTok{li}\OperatorTok{>}\NormalTok{One Item}\OperatorTok{<}\NormalTok{/li}\OperatorTok{>}        
            \OperatorTok{<}\ExtensionTok{li}\OperatorTok{>}\NormalTok{Another Item}\OperatorTok{<}\NormalTok{/li}\OperatorTok{>}    
            \OperatorTok{<}\ExtensionTok{li}\OperatorTok{>}\NormalTok{Yet Another Item}\OperatorTok{<}\NormalTok{/li}\OperatorTok{>}
        \OperatorTok{<}\NormalTok{/}\ExtensionTok{ul}\OperatorTok{>}                        
    \OperatorTok{<}\NormalTok{/}\ExtensionTok{body}\OperatorTok{>}             
\OperatorTok{<}\NormalTok{/}\ExtensionTok{html}\OperatorTok{>}                 
\end{Highlighting}
\end{Shaded}

Ordered lists number lines in increasing order.

\begin{Shaded}
\begin{Highlighting}[]
\OperatorTok{<}\NormalTok{!}\ExtensionTok{DOCTYPE}\NormalTok{ html}\OperatorTok{>}         
\OperatorTok{<}\ExtensionTok{html}\OperatorTok{>}                  
    \OperatorTok{<}\ExtensionTok{body}\OperatorTok{>}              
        \ExtensionTok{An}\NormalTok{ Ordered List:}
        \OperatorTok{<}\ExtensionTok{ol}\OperatorTok{>}                          
            \OperatorTok{<}\ExtensionTok{li}\OperatorTok{>}\NormalTok{First Item}\OperatorTok{<}\NormalTok{/li}\OperatorTok{>}       
            \OperatorTok{<}\ExtensionTok{li}\OperatorTok{>}\NormalTok{Second Item}\OperatorTok{<}\NormalTok{/li}\OperatorTok{>}      
            \OperatorTok{<}\ExtensionTok{li}\OperatorTok{>}\NormalTok{Another Item Here}\OperatorTok{<}\NormalTok{/li}\OperatorTok{>}
            \OperatorTok{<}\ExtensionTok{li}\OperatorTok{>}\NormalTok{Fourth Item}\OperatorTok{<}\NormalTok{/li}\OperatorTok{>}      
        \OperatorTok{<}\NormalTok{/}\ExtensionTok{ol}\OperatorTok{>}                         
    \OperatorTok{<}\NormalTok{/}\ExtensionTok{body}\OperatorTok{>}             
\OperatorTok{<}\NormalTok{/}\ExtensionTok{html}\OperatorTok{>}                 
\end{Highlighting}
\end{Shaded}

The image tag refers to and inserts an image as an html attribute. The
alt gives alternative code if the image is missing. The height and width
sets the image size in number of pixels. When the image size is set to
50\% sets the image size dynamically to 50\% of the browser width or
height.

\begin{Shaded}
\begin{Highlighting}[]
\OperatorTok{<}\NormalTok{!}\ExtensionTok{DOCTYPE}\NormalTok{ html}\OperatorTok{>}         
\OperatorTok{<}\ExtensionTok{html}\OperatorTok{>}                  
    \OperatorTok{<}\ExtensionTok{body}\OperatorTok{>}              
        \OperatorTok{<}\ExtensionTok{img}\NormalTok{ src=}\StringTok{"cat.jpg"}\NormalTok{ alt=}\StringTok{"cat"}\NormalTok{ width=}\StringTok{"300"}\NormalTok{ height=}\StringTok{"200"}\OperatorTok{>}
        \OperatorTok{<}\ExtensionTok{img}\NormalTok{ src=}\StringTok{"cat.jpg"}\NormalTok{ alt=}\StringTok{"cat"}\NormalTok{ width=}\StringTok{"50%"} \OperatorTok{>}
    \OperatorTok{<}\NormalTok{/}\ExtensionTok{body}\OperatorTok{>}             
\OperatorTok{<}\NormalTok{/}\ExtensionTok{html}\OperatorTok{>}                 
\end{Highlighting}
\end{Shaded}

Tables display data in a table format that can be styled in various
ways. The \texttt{th} tag specifies the headings of each of the columns.
The \texttt{td} tag specifies the data in each of the columns.

\begin{Shaded}
\begin{Highlighting}[]
\OperatorTok{<}\NormalTok{!}\ExtensionTok{DOCTYPE}\NormalTok{ html}\OperatorTok{>}         
\OperatorTok{<}\ExtensionTok{html}\OperatorTok{>}                  
    \OperatorTok{<}\ExtensionTok{body}\OperatorTok{>}              
        \OperatorTok{<}\ExtensionTok{table}\OperatorTok{>}                         
            \OperatorTok{<}\FunctionTok{tr}\OperatorTok{>}                        
                \OperatorTok{<}\ExtensionTok{th}\OperatorTok{>}\NormalTok{First Name}\OperatorTok{<}\NormalTok{/th}\OperatorTok{>}     
                \OperatorTok{<}\ExtensionTok{th}\OperatorTok{>}\NormalTok{Last Name}\OperatorTok{<}\NormalTok{/th}\OperatorTok{>}      
                \OperatorTok{<}\ExtensionTok{th}\OperatorTok{>}\NormalTok{Years in Office}\OperatorTok{<}\NormalTok{/th}\OperatorTok{>}
            \OperatorTok{<}\NormalTok{/}\ExtensionTok{tr}\OperatorTok{>}                       
            \OperatorTok{<}\FunctionTok{tr}\OperatorTok{>}                        
                \OperatorTok{<}\ExtensionTok{td}\OperatorTok{>}\NormalTok{George}\OperatorTok{<}\NormalTok{/td}\OperatorTok{>}         
                \OperatorTok{<}\ExtensionTok{td}\OperatorTok{>}\NormalTok{Washington}\OperatorTok{<}\NormalTok{/td}\OperatorTok{>}     
                \OperatorTok{<}\ExtensionTok{td}\OperatorTok{>}\NormalTok{1789-}\OperatorTok{1797<}\NormalTok{/td}\OperatorTok{>}      
            \OperatorTok{<}\NormalTok{/}\ExtensionTok{tr}\OperatorTok{>}                       
            \OperatorTok{<}\FunctionTok{tr}\OperatorTok{>}                        
                \OperatorTok{<}\ExtensionTok{td}\OperatorTok{>}\NormalTok{John}\OperatorTok{<}\NormalTok{/td}\OperatorTok{>}           
                \OperatorTok{<}\ExtensionTok{td}\OperatorTok{>}\NormalTok{Adams}\OperatorTok{<}\NormalTok{/td}\OperatorTok{>}          
                \OperatorTok{<}\ExtensionTok{td}\OperatorTok{>}\NormalTok{1797-}\OperatorTok{1801<}\NormalTok{/td}\OperatorTok{>}      
            \OperatorTok{<}\NormalTok{/}\ExtensionTok{tr}\OperatorTok{>}                       
            \OperatorTok{<}\FunctionTok{tr}\OperatorTok{>}                        
                \OperatorTok{<}\ExtensionTok{td}\OperatorTok{>}\NormalTok{Thomas}\OperatorTok{<}\NormalTok{/td}\OperatorTok{>}         
                \OperatorTok{<}\ExtensionTok{td}\OperatorTok{>}\NormalTok{Jefferson}\OperatorTok{<}\NormalTok{/td}\OperatorTok{>}      
                \OperatorTok{<}\ExtensionTok{td}\OperatorTok{>}\NormalTok{1801-}\OperatorTok{1809<}\NormalTok{/td}\OperatorTok{>}      
            \OperatorTok{<}\NormalTok{/}\ExtensionTok{tr}\OperatorTok{>}                       
        \OperatorTok{<}\NormalTok{/}\ExtensionTok{table}\OperatorTok{>}                        
    \OperatorTok{<}\NormalTok{/}\ExtensionTok{body}\OperatorTok{>}             
\OperatorTok{<}\NormalTok{/}\ExtensionTok{html}\OperatorTok{>}                 
\end{Highlighting}
\end{Shaded}

Tables can be styled within the header of the html document. Both the
\texttt{th} and the \texttt{td} styles are defined together.
\texttt{border-collapse} combines the borders of cells together.

\begin{Shaded}
\begin{Highlighting}[]
\OperatorTok{<}\NormalTok{!}\ExtensionTok{DOCTYPE}\NormalTok{ html}\OperatorTok{>}         
\OperatorTok{<}\ExtensionTok{html}\OperatorTok{>}                  
    \OperatorTok{<}\FunctionTok{head}\OperatorTok{>}                                  
        \OperatorTok{<}\ExtensionTok{title}\OperatorTok{>}\NormalTok{Presidents}\OperatorTok{<}\NormalTok{/title}\OperatorTok{>}           
        \OperatorTok{<}\ExtensionTok{style}\OperatorTok{>}                             
            \ExtensionTok{table}\NormalTok{ \{                         }
                \ExtensionTok{border}\NormalTok{: 2px solid black}\KeywordTok{;}    
                \ExtensionTok{border-collapse}\NormalTok{: collapse}\KeywordTok{;}  
                \ExtensionTok{width}\NormalTok{: 50%}\KeywordTok{;}                 
\NormalTok{            \}                               }
                                            
            \ExtensionTok{th}\NormalTok{, td \{                        }
                \ExtensionTok{border}\NormalTok{: 1px solid black}\KeywordTok{;}    
                \ExtensionTok{padding}\NormalTok{: 5px}\KeywordTok{;}               
                \ExtensionTok{text-align}\NormalTok{: center}\KeywordTok{;}         
\NormalTok{            \}                               }
                                            
            \ExtensionTok{th}\NormalTok{ \{                            }
                \ExtensionTok{background-color}\NormalTok{: lightgray}\KeywordTok{;}
\NormalTok{            \}                               }
        \OperatorTok{<}\NormalTok{/}\ExtensionTok{style}\OperatorTok{>}                            
    \OperatorTok{<}\NormalTok{/}\ExtensionTok{head}\OperatorTok{>}                                 
    \OperatorTok{<}\ExtensionTok{body}\OperatorTok{>}              
        \OperatorTok{<}\ExtensionTok{table}\OperatorTok{>}                         
            \OperatorTok{<}\FunctionTok{tr}\OperatorTok{>}                        
                \OperatorTok{<}\ExtensionTok{th}\OperatorTok{>}\NormalTok{First Name}\OperatorTok{<}\NormalTok{/th}\OperatorTok{>}     
                \OperatorTok{<}\ExtensionTok{th}\OperatorTok{>}\NormalTok{Last Name}\OperatorTok{<}\NormalTok{/th}\OperatorTok{>}      
                \OperatorTok{<}\ExtensionTok{th}\OperatorTok{>}\NormalTok{Years in Office}\OperatorTok{<}\NormalTok{/th}\OperatorTok{>}
            \OperatorTok{<}\NormalTok{/}\ExtensionTok{tr}\OperatorTok{>}                       
            \OperatorTok{<}\FunctionTok{tr}\OperatorTok{>}                        
                \OperatorTok{<}\ExtensionTok{td}\OperatorTok{>}\NormalTok{George}\OperatorTok{<}\NormalTok{/td}\OperatorTok{>}         
                \OperatorTok{<}\ExtensionTok{td}\OperatorTok{>}\NormalTok{Washington}\OperatorTok{<}\NormalTok{/td}\OperatorTok{>}     
                \OperatorTok{<}\ExtensionTok{td}\OperatorTok{>}\NormalTok{1789-}\OperatorTok{1797<}\NormalTok{/td}\OperatorTok{>}      
            \OperatorTok{<}\NormalTok{/}\ExtensionTok{tr}\OperatorTok{>}                       
        \OperatorTok{<}\NormalTok{/}\ExtensionTok{table}\OperatorTok{>}                        
    \OperatorTok{<}\NormalTok{/}\ExtensionTok{body}\OperatorTok{>}             
\OperatorTok{<}\NormalTok{/}\ExtensionTok{html}\OperatorTok{>}                 
\end{Highlighting}
\end{Shaded}

Forms can be created and labeled as such. The \texttt{placeholder} text
is what is written within the form before anything is entered into it.
The \texttt{name} is similar to a variable name and can be used to refer
to the form and the data that is entered into it. The text within the
button is the text that will appear on the button in the page.

\begin{Shaded}
\begin{Highlighting}[]
\OperatorTok{<}\NormalTok{!}\ExtensionTok{DOCTYPE}\NormalTok{ html}\OperatorTok{>}         
\OperatorTok{<}\ExtensionTok{html}\OperatorTok{>}                  
    \OperatorTok{<}\ExtensionTok{body}\OperatorTok{>}              
        \OperatorTok{<}\ExtensionTok{form}\OperatorTok{>}                                                     
            \OperatorTok{<}\ExtensionTok{input}\NormalTok{ type=}\StringTok{"text"}\NormalTok{ placeholder=}\StringTok{"Full Name"}\NormalTok{ name=}\StringTok{"name"}\OperatorTok{>}
            \OperatorTok{<}\ExtensionTok{button}\OperatorTok{>}\NormalTok{Submit!}\OperatorTok{<}\NormalTok{/button}\OperatorTok{>}                               
        \OperatorTok{<}\NormalTok{/}\ExtensionTok{form}\OperatorTok{>}                                                    
    \OperatorTok{<}\NormalTok{/}\ExtensionTok{body}\OperatorTok{>}             
\OperatorTok{<}\NormalTok{/}\ExtensionTok{html}\OperatorTok{>}                 
\end{Highlighting}
\end{Shaded}

Text can be aligned and colord by specifying styles within the
respective tags of text.

\begin{Shaded}
\begin{Highlighting}[]
\OperatorTok{<}\NormalTok{!}\ExtensionTok{DOCTYPE}\NormalTok{ html}\OperatorTok{>}         
\OperatorTok{<}\ExtensionTok{html}\OperatorTok{>}                  
    \OperatorTok{<}\ExtensionTok{body}\OperatorTok{>}              
        \OperatorTok{<}\ExtensionTok{h1}\NormalTok{ style=}\StringTok{"color:red;text-align:center;"}\OperatorTok{>}\NormalTok{Welcome to My Web Page!}\OperatorTok{<}\NormalTok{/h1}\OperatorTok{>}
        \OperatorTok{<}\ExtensionTok{h1}\NormalTok{ style=}\StringTok{"color:#4290f5;text-align:center;"}\OperatorTok{>}\NormalTok{Second heading}\OperatorTok{<}\NormalTok{/h1}\OperatorTok{>}
    \OperatorTok{<}\NormalTok{/}\ExtensionTok{body}\OperatorTok{>}             
\OperatorTok{<}\NormalTok{/}\ExtensionTok{html}\OperatorTok{>}                 
\end{Highlighting}
\end{Shaded}

Style elements can be separated from the actual body of the webpage. In
this example every \texttt{h1} is styled within the style portion of the
header.

\begin{Shaded}
\begin{Highlighting}[]
\OperatorTok{<}\NormalTok{!}\ExtensionTok{DOCTYPE}\NormalTok{ html}\OperatorTok{>}         
\OperatorTok{<}\ExtensionTok{html}\OperatorTok{>}                  
    \OperatorTok{<}\FunctionTok{head}\OperatorTok{>}                         
        \OperatorTok{<}\ExtensionTok{title}\OperatorTok{>}\NormalTok{My Web Page!}\OperatorTok{<}\NormalTok{/title}\OperatorTok{>}
        \OperatorTok{<}\ExtensionTok{style}\OperatorTok{>}                    
            \ExtensionTok{h1}\NormalTok{ \{                   }
                \ExtensionTok{color}\NormalTok{: red}\KeywordTok{;}        
                \ExtensionTok{text-align}\NormalTok{: center}\KeywordTok{;}
\NormalTok{            \}                      }
        \OperatorTok{<}\NormalTok{/}\ExtensionTok{style}\OperatorTok{>}                   
    \OperatorTok{<}\NormalTok{/}\ExtensionTok{head}\OperatorTok{>}                        
    \OperatorTok{<}\ExtensionTok{body}\OperatorTok{>}              
        \OperatorTok{<}\ExtensionTok{h1}\OperatorTok{>}\NormalTok{Welcome to My Web Page!}\OperatorTok{<}\NormalTok{/h1}\OperatorTok{>}
    \OperatorTok{<}\NormalTok{/}\ExtensionTok{body}\OperatorTok{>}             
\OperatorTok{<}\NormalTok{/}\ExtensionTok{html}\OperatorTok{>}                 
\end{Highlighting}
\end{Shaded}

Links to local pages or hyperlinks are included within the
\texttt{\textless{}a\textgreater{}} tag.

\begin{Shaded}
\begin{Highlighting}[]
\OperatorTok{<}\ExtensionTok{a}\NormalTok{ href=}\StringTok{"about.html"}\OperatorTok{>}\NormalTok{About}\OperatorTok{<}\NormalTok{/a}\OperatorTok{>}
\OperatorTok{<}\ExtensionTok{a}\NormalTok{ href=}\StringTok{"http://www.google.com"}\OperatorTok{>}\NormalTok{Google}\OperatorTok{<}\NormalTok{/a}\OperatorTok{>}
\end{Highlighting}
\end{Shaded}

Links can also refer to locations on the same page.

\begin{Shaded}
\begin{Highlighting}[]
\OperatorTok{<}\ExtensionTok{a}\NormalTok{ href=}\StringTok{"section1"}\OperatorTok{>}\NormalTok{Section }\OperatorTok{1<}\NormalTok{/a}\OperatorTok{>}
\OperatorTok{<}\ExtensionTok{h1}\NormalTok{ id=}\StringTok{"section1"}\OperatorTok{>}\NormalTok{Some stuff.}\OperatorTok{<}\NormalTok{/h1}\OperatorTok{>}
\end{Highlighting}
\end{Shaded}

A newline can be inserted within the body of text by using
\texttt{\textless{}br\ /\textgreater{}}.

\begin{Shaded}
\begin{Highlighting}[]
\OperatorTok{<}\ExtensionTok{a}\NormalTok{ href=}\StringTok{"about.html"}\OperatorTok{>}\NormalTok{About}\OperatorTok{<}\NormalTok{/a}\OperatorTok{><}\NormalTok{br /}\OperatorTok{>}
\end{Highlighting}
\end{Shaded}

\subsection{Forms}\label{forms}

Generic text input fields can be created with the text type.

\begin{Shaded}
\begin{Highlighting}[]
\OperatorTok{<}\ExtensionTok{div}\OperatorTok{>}
\OperatorTok{<}\ExtensionTok{input}\NormalTok{ name=}\StringTok{"name"}\NormalTok{ type=}\StringTok{"text"}\NormalTok{ placeholder=}\StringTok{"Name"}\OperatorTok{>}
\OperatorTok{<}\NormalTok{/}\ExtensionTok{div}\OperatorTok{>}
\end{Highlighting}
\end{Shaded}

A password field is pretty similar to a text field, but the characters
are obscured.

\begin{Shaded}
\begin{Highlighting}[]
\OperatorTok{<}\ExtensionTok{input}\NormalTok{ name=}\StringTok{"password"}\NormalTok{ type=}\StringTok{"password"}\NormalTok{ placeholder=}\StringTok{"Password"}\OperatorTok{>}
\end{Highlighting}
\end{Shaded}

Dropdown lists of the possible valid choices for a field can be used
with \texttt{datalist}.

\begin{Shaded}
\begin{Highlighting}[]
\OperatorTok{<}\ExtensionTok{input}\NormalTok{ name=}\StringTok{"country"}\NormalTok{ list=}\StringTok{"countries"}\NormalTok{ placeholder=}\StringTok{"Country"}\OperatorTok{>}
\OperatorTok{<}\ExtensionTok{datalist}\NormalTok{ id=}\StringTok{"countries"}\OperatorTok{>}
    \OperatorTok{<}\ExtensionTok{option}\NormalTok{ value=}\StringTok{"Afghanistan"}\OperatorTok{>}
    \OperatorTok{<}\ExtensionTok{option}\NormalTok{ value=}\StringTok{"Albania"}\OperatorTok{>}
    \OperatorTok{<}\ExtensionTok{option}\NormalTok{ value=}\StringTok{"Algeria"}\OperatorTok{>}
\end{Highlighting}
\end{Shaded}

\section{CSS}\label{css}

Commenting in CSS.

\begin{Shaded}
\begin{Highlighting}[]
\ExtensionTok{/*}
\ExtensionTok{These}\NormalTok{ are some comments.}
\ExtensionTok{*/}
\end{Highlighting}
\end{Shaded}

CSS properties can be found
\href{https://developer.mozilla.org/en-US/docs/Web/CSS/Reference}{here}.

Instead of putting the css styles within the header of the html file,
they can be included in a separate css file and referenced. In this
example, the type of file being referenced is classified as a
\texttt{stylesheet} and the code is within \texttt{styles.css}.

\begin{Shaded}
\begin{Highlighting}[]
\OperatorTok{<}\NormalTok{!}\ExtensionTok{DOCTYPE}\NormalTok{ html}\OperatorTok{>}         
\OperatorTok{<}\ExtensionTok{html}\OperatorTok{>}                  
    \OperatorTok{<}\FunctionTok{head}\OperatorTok{>}                                       
        \OperatorTok{<}\FunctionTok{link}\NormalTok{ rel=}\StringTok{"stylesheet"}\NormalTok{ href=}\StringTok{"styles.css"}\OperatorTok{>}
    \OperatorTok{<}\NormalTok{/}\ExtensionTok{head}\OperatorTok{>}                                      
\OperatorTok{<}\NormalTok{/}\ExtensionTok{html}\OperatorTok{>}                 
\end{Highlighting}
\end{Shaded}

The code that goes within the css file is here, and it is simply the
same code that was put into the style headers in the above example.

\begin{Shaded}
\begin{Highlighting}[]
\ExtensionTok{h1}\NormalTok{ \{                   }
    \ExtensionTok{color}\NormalTok{: blue}\KeywordTok{;}       
    \ExtensionTok{text-align}\NormalTok{: center}\KeywordTok{;}
\NormalTok{\}                      }
\end{Highlighting}
\end{Shaded}

Divisions define sections of the code that can be separated so it can be
controlled in a particular manner. Font priorities are taken left to
right if some fonts are not found.

\begin{Shaded}
\begin{Highlighting}[]
\OperatorTok{<}\NormalTok{!}\ExtensionTok{DOCTYPE}\NormalTok{ html}\OperatorTok{>}                        
\OperatorTok{<}\ExtensionTok{html}\OperatorTok{>}                                 
    \OperatorTok{<}\FunctionTok{head}\OperatorTok{>}                             
        \OperatorTok{<}\ExtensionTok{title}\OperatorTok{>}\NormalTok{My Web Page!}\OperatorTok{<}\NormalTok{/title}\OperatorTok{>}    
        \OperatorTok{<}\ExtensionTok{style}\OperatorTok{>}                        
            \ExtensionTok{div}\NormalTok{ \{                      }
                \ExtensionTok{background-color}\NormalTok{: teal}\KeywordTok{;}
                \ExtensionTok{width}\NormalTok{: 500px}\KeywordTok{;}          
                \ExtensionTok{height}\NormalTok{: 400px}\KeywordTok{;}         
                \ExtensionTok{margin}\NormalTok{: 30px}\KeywordTok{;}
                \ExtensionTok{padding}\NormalTok{: 20px}\KeywordTok{;}
                \ExtensionTok{font-family}\NormalTok{: Arial, sans-serif}\KeywordTok{;}
                \ExtensionTok{font-size}\NormalTok{: 28px}\KeywordTok{;}               
                \ExtensionTok{font-weight}\NormalTok{: bold}\KeywordTok{;}             
                \ExtensionTok{border}\NormalTok{: 1px dotted black}\KeywordTok{;}
\NormalTok{            \}                          }
        \OperatorTok{<}\NormalTok{/}\ExtensionTok{style}\OperatorTok{>}                       
    \OperatorTok{<}\NormalTok{/}\ExtensionTok{head}\OperatorTok{>}                            
    \OperatorTok{<}\ExtensionTok{body}\OperatorTok{>}                             
        \OperatorTok{<}\ExtensionTok{div}\OperatorTok{>}                          
            \ExtensionTok{Hello}\NormalTok{, world!              }
        \OperatorTok{<}\NormalTok{/}\ExtensionTok{div}\OperatorTok{>}                         
    \OperatorTok{<}\NormalTok{/}\ExtensionTok{body}\OperatorTok{>}                            
\OperatorTok{<}\NormalTok{/}\ExtensionTok{html}\OperatorTok{>}                                
\end{Highlighting}
\end{Shaded}

Divisions and spans can be named and used to refer to different parts of
the html document specifically.

\begin{Shaded}
\begin{Highlighting}[]
\OperatorTok{<}\NormalTok{!}\ExtensionTok{DOCTYPE}\NormalTok{ html}\OperatorTok{>}                                                         
\OperatorTok{<}\ExtensionTok{html}\OperatorTok{>}                                                                  
    \OperatorTok{<}\FunctionTok{head}\OperatorTok{>}                                                              
        \OperatorTok{<}\ExtensionTok{title}\OperatorTok{>}\NormalTok{My Web Page!}\OperatorTok{<}\NormalTok{/title}\OperatorTok{>}                                     
        \OperatorTok{<}\ExtensionTok{style}\OperatorTok{>}                                                         
            \CommentTok{#top \{                                                      }
                \ExtensionTok{font-size}\NormalTok{: 36px}\KeywordTok{;}                                        
                \ExtensionTok{color}\NormalTok{: red}\KeywordTok{;}                                             
\NormalTok{            \}                                                           }
                                                                        
            \ExtensionTok{.name}\NormalTok{ \{                                                     }
                \ExtensionTok{font-weight}\NormalTok{: bold}\KeywordTok{;}                                      
                \ExtensionTok{color}\NormalTok{: blue}\KeywordTok{;}                                            
\NormalTok{            \}                                                           }
        \OperatorTok{<}\NormalTok{/}\ExtensionTok{style}\OperatorTok{>}                                                        
    \OperatorTok{<}\NormalTok{/}\ExtensionTok{head}\OperatorTok{>}                                                             
    \OperatorTok{<}\ExtensionTok{body}\OperatorTok{>}                                                              
        \OperatorTok{<}\ExtensionTok{div}\NormalTok{ id=}\StringTok{"top"}\OperatorTok{>}                                                  
            \ExtensionTok{This}\NormalTok{ is the }\OperatorTok{<}\NormalTok{span class=}\StringTok{"name"}\OperatorTok{>}\NormalTok{top}\OperatorTok{<}\NormalTok{/span}\OperatorTok{>}\NormalTok{ of my web page.   }
        \OperatorTok{<}\NormalTok{/}\ExtensionTok{div}\OperatorTok{>}                                                          
    \OperatorTok{<}\NormalTok{/}\ExtensionTok{body}\OperatorTok{>}                                                             
\OperatorTok{<}\NormalTok{/}\ExtensionTok{html}\OperatorTok{>}
\end{Highlighting}
\end{Shaded}

Link styling can be done by adjusting colors and text decorations of
links.

\begin{Shaded}
\begin{Highlighting}[]
\OperatorTok{<}\ExtensionTok{style}\OperatorTok{>}                             
    \ExtensionTok{a}\NormalTok{:link \{                        }
      \ExtensionTok{color}\NormalTok{: blue}\KeywordTok{;}                  
      \ExtensionTok{background-color}\NormalTok{: transparent}\KeywordTok{;}
      \ExtensionTok{text-decoration}\NormalTok{: none}\KeywordTok{;}        
\NormalTok{    \}                               }
                                    
    \ExtensionTok{a}\NormalTok{:visited \{                     }
      \ExtensionTok{color}\NormalTok{: red}\KeywordTok{;}                   
      \ExtensionTok{background-color}\NormalTok{: transparent}\KeywordTok{;}
      \ExtensionTok{text-decoration}\NormalTok{: none}\KeywordTok{;}        
\NormalTok{    \}                               }
                                    
    \ExtensionTok{a}\NormalTok{:hover \{                       }
      \ExtensionTok{color}\NormalTok{: pink}\KeywordTok{;}                  
      \ExtensionTok{background-color}\NormalTok{: transparent}\KeywordTok{;}
      \ExtensionTok{text-decoration}\NormalTok{: underline}\KeywordTok{;}   
\NormalTok{    \}                               }
                                    
    \ExtensionTok{a}\NormalTok{:active \{                      }
      \ExtensionTok{color}\NormalTok{: orange}\KeywordTok{;}                
      \ExtensionTok{background-color}\NormalTok{: transparent}\KeywordTok{;}
      \ExtensionTok{text-decoration}\NormalTok{: underline}\KeywordTok{;}   
\NormalTok{    \}                               }
\OperatorTok{<}\NormalTok{/}\ExtensionTok{style}\OperatorTok{>}                            
\end{Highlighting}
\end{Shaded}

Fonts can be imported from locations like google's hosted fonts and used
directly to avoid problems with a browser not supporting them. The link
to the fonts goes within the header portion of the html code.

\begin{Shaded}
\begin{Highlighting}[]
\OperatorTok{<}\FunctionTok{head}\OperatorTok{>}
    \OperatorTok{<}\FunctionTok{link}\NormalTok{ href=}\StringTok{"https://fonts.googleapis.com/css?family=Cormorant+Garamond|Proza+Libre&display=swap"}\NormalTok{ rel=}\StringTok{"stylesheet"}\OperatorTok{>}
\OperatorTok{<}\NormalTok{/}\ExtensionTok{head}\OperatorTok{>}
\end{Highlighting}
\end{Shaded}

The imported font families can then be used within the CSS directly.

\begin{Shaded}
\begin{Highlighting}[]
\ExtensionTok{body}\NormalTok{ \{}
    \ExtensionTok{font-family}\NormalTok{: Proza Libre, Cormorant Garamond}
\NormalTok{\}}
\end{Highlighting}
\end{Shaded}

Nested elements can be styled in a grouped manner.

\begin{Shaded}
\begin{Highlighting}[]
\OperatorTok{<}\ExtensionTok{style}\OperatorTok{>}
    \ExtensionTok{ol}\NormalTok{ li \{}
        \ExtensionTok{color}\NormalTok{: red}\KeywordTok{;}
\NormalTok{    \}}
\OperatorTok{<}\NormalTok{/}\ExtensionTok{style}\OperatorTok{>}
\OperatorTok{<}\ExtensionTok{body}\OperatorTok{>}
    \OperatorTok{<}\ExtensionTok{ol}\OperatorTok{>}
        \OperatorTok{<}\ExtensionTok{li}\OperatorTok{>}\NormalTok{list item}\OperatorTok{<}\NormalTok{/li}\OperatorTok{>}
        \OperatorTok{<}\ExtensionTok{li}\OperatorTok{>}\NormalTok{second list item}\OperatorTok{<}\NormalTok{/li}\OperatorTok{>}
    \OperatorTok{<}\NormalTok{/}\ExtensionTok{ol}\OperatorTok{>}
\OperatorTok{<}\NormalTok{/}\ExtensionTok{body}\OperatorTok{>}
\end{Highlighting}
\end{Shaded}

A similar use is to style the immediately nested child elements and none
other using the \texttt{\textgreater{}} operator.

\begin{Shaded}
\begin{Highlighting}[]
\OperatorTok{<}\ExtensionTok{style}\OperatorTok{>}
    \ExtensionTok{ol} \OperatorTok{>}\NormalTok{ li \{}
        \ExtensionTok{color}\NormalTok{: red}\KeywordTok{;}
\NormalTok{    \}}
\OperatorTok{<}\NormalTok{/}\ExtensionTok{style}\OperatorTok{>}
\OperatorTok{<}\ExtensionTok{body}\OperatorTok{>}
    \OperatorTok{<}\ExtensionTok{ol}\OperatorTok{>}
        \OperatorTok{<}\ExtensionTok{li}\OperatorTok{>}\NormalTok{this will be colored}\OperatorTok{<}\NormalTok{/li}\OperatorTok{>}
        \OperatorTok{<}\ExtensionTok{ul}\OperatorTok{>}
            \OperatorTok{<}\ExtensionTok{li}\OperatorTok{>}\NormalTok{this won}\StringTok{'t be colored</li>}
\StringTok{        </ul>}
\StringTok{        <li>this will also be colored</li>}
\StringTok{    </ol>}
\StringTok{</body>}
\end{Highlighting}
\end{Shaded}

Fields of particular types can be styled based on their type. These are
examples of a text field that allows letters and numbers and a number
field that only allows numbers.\\
The fields can then be styled based on the type of field that they are.

\begin{Shaded}
\begin{Highlighting}[]
\OperatorTok{<}\ExtensionTok{style}\OperatorTok{>}
    \ExtensionTok{input}\NormalTok{[type=text] \{}
    \ExtensionTok{background-color}\NormalTok{: red}\KeywordTok{;}
\NormalTok{\}}
\OperatorTok{<}\NormalTok{/}\ExtensionTok{style}\OperatorTok{>}

\OperatorTok{<}\ExtensionTok{body}\OperatorTok{>}
    \OperatorTok{<}\ExtensionTok{input}\NormalTok{ name=}\StringTok{"name"}\NormalTok{ type=}\StringTok{"text"}\NormalTok{ placeholder=}\StringTok{"First Name"}\OperatorTok{>}
    \OperatorTok{<}\ExtensionTok{input}\NormalTok{ name=}\StringTok{"name"}\NormalTok{ type=}\StringTok{"number"}\NormalTok{ placeholder=}\StringTok{"Age"}\OperatorTok{>}
\OperatorTok{<}\NormalTok{/}\ExtensionTok{body}\OperatorTok{>}
\end{Highlighting}
\end{Shaded}

\subsection{Selectors}\label{selectors}

CSS Selectors allow specific classes or elements to be selected and
styled individually.

\begin{Shaded}
\begin{Highlighting}[]
\ExtensionTok{a}\NormalTok{, b /* Multiple element selector */}
\ExtensionTok{a}\NormalTok{ b /* Descendant selector */}
\ExtensionTok{a} \OperatorTok{>}\NormalTok{ b /* Child selector */}
\ExtensionTok{a}\NormalTok{ + b /* Adjacent sibling selector */}
\NormalTok{[}\VariableTok{a=}\NormalTok{b] }\ExtensionTok{/*}\NormalTok{ Attribute selector */}
\ExtensionTok{a}\NormalTok{:b /* Pseudoclass selector */}
\ExtensionTok{a}\NormalTok{::b /* Pseudoelement selector */}
\end{Highlighting}
\end{Shaded}

Pseudo-classes allow for different styling effects depending on the
state of the element.

\begin{Shaded}
\begin{Highlighting}[]
\OperatorTok{<}\ExtensionTok{style}\OperatorTok{>}
    \ExtensionTok{button}\NormalTok{ \{}
        \ExtensionTok{background-color}\NormalTok{: green}\KeywordTok{;}
\NormalTok{    \}}
    \ExtensionTok{button}\NormalTok{:hover \{}
        \ExtensionTok{background-color}\NormalTok{: orange}\KeywordTok{;}
\NormalTok{    \}}
\OperatorTok{<}\NormalTok{/}\ExtensionTok{style}\OperatorTok{>}
\OperatorTok{<}\ExtensionTok{body}\OperatorTok{>}
    \OperatorTok{<}\ExtensionTok{button}\OperatorTok{>}\NormalTok{Click}\OperatorTok{<}\NormalTok{/button}\OperatorTok{>}
\OperatorTok{<}\NormalTok{/}\ExtensionTok{body}\OperatorTok{>}
\end{Highlighting}
\end{Shaded}

Pseudo-elements are similar but select elements also allow things to be
styled by placing information at the beginning of an item.\\
What is happening here is that there is a link amd before the link it
says ``Click here:'', and to the left of that the
\texttt{\textbackslash{}21d12} specifies a unicode arrow.

\begin{Shaded}
\begin{Highlighting}[]
\OperatorTok{<}\ExtensionTok{style}\OperatorTok{>}
    \ExtensionTok{a}\NormalTok{::before \{}
        \ExtensionTok{content}\NormalTok{: }\StringTok{"\textbackslash{}21d2 Click here: "}\KeywordTok{;}
        \ExtensionTok{font-weight}\NormalTok{: bold}\KeywordTok{;}
\NormalTok{    \}}
\OperatorTok{<}\NormalTok{/}\ExtensionTok{style}\OperatorTok{>}
\OperatorTok{<}\ExtensionTok{body}\OperatorTok{>}
    \OperatorTok{<}\ExtensionTok{a}\NormalTok{ href=}\StringTok{"#"}\OperatorTok{>}\NormalTok{A link}\OperatorTok{<}\NormalTok{/a}\OperatorTok{>}
\OperatorTok{<}\NormalTok{/}\ExtensionTok{body}\OperatorTok{>}
\end{Highlighting}
\end{Shaded}

Text highlighting can also be controlled with pseudo-elements. Here the
text color turns red and the highlight is in yellow when the text is
highlighted.

\begin{Shaded}
\begin{Highlighting}[]
\OperatorTok{<}\ExtensionTok{style}\OperatorTok{>}
    \ExtensionTok{p}\NormalTok{::selection \{}
        \ExtensionTok{color}\NormalTok{: red}\KeywordTok{;}
        \ExtensionTok{background-color}\NormalTok{: yellow}\KeywordTok{;}
\NormalTok{    \}}
\OperatorTok{<}\NormalTok{/}\ExtensionTok{style}\OperatorTok{>}
\OperatorTok{<}\ExtensionTok{body}\OperatorTok{>}
    \OperatorTok{<}\ExtensionTok{p}\OperatorTok{>}\NormalTok{This is some text}\OperatorTok{<}\NormalTok{/p}\OperatorTok{>}
\OperatorTok{<}\NormalTok{/}\ExtensionTok{body}\OperatorTok{>}
\end{Highlighting}
\end{Shaded}

\subsection{Responsive Design}\label{responsive-design}

Media queries are CSS rules that are only used if certain properties are
true. A commonly used property is screen size to adjust page layouts
fore mobile. It is generally a good idea to design for mobile first and
adjust properties to fit desktop, as this will ensure mobile gets the
fastest performace.

Here the width of a column is being altered if the browser window is at
least 768px in size. Altering the design in this way illustrates how
development can be done ``mobile-first'', as the property is altered if
a desktop is used instead of mobile.

\begin{Shaded}
\begin{Highlighting}[]
\ExtensionTok{@media}\NormalTok{ only screen and (min-width: 768px) }\KeywordTok{\{}
  \ExtensionTok{/*}\NormalTok{ For desktop: */}
  \ExtensionTok{.col-1}\NormalTok{ \{width: 8.33%}\KeywordTok{;\}}
\NormalTok{\}}
\end{Highlighting}
\end{Shaded}

Media queries can also control content what content gets printed. Here
both paragraphs get displayed on the page, but only the first paragraph
appears when the page is printed.

\begin{Shaded}
\begin{Highlighting}[]
\OperatorTok{<}\ExtensionTok{style}\OperatorTok{>}
    \ExtensionTok{@media}\NormalTok{ print \{}
        \ExtensionTok{.screen-only}\NormalTok{ \{}
            \ExtensionTok{display}\NormalTok{: none}\KeywordTok{;}
\NormalTok{        \}}
\NormalTok{    \}}
\OperatorTok{<}\NormalTok{/}\ExtensionTok{style}\OperatorTok{>}
\OperatorTok{<}\ExtensionTok{body}\OperatorTok{>}
    \OperatorTok{<}\ExtensionTok{p}\OperatorTok{>}\NormalTok{This gets printed}\OperatorTok{<}\NormalTok{/p}\OperatorTok{>}
    \OperatorTok{<}\ExtensionTok{p}\NormalTok{ class=}\StringTok{"screen-only"}\OperatorTok{>}\NormalTok{This does not get printed}\OperatorTok{<}\NormalTok{/p}\OperatorTok{>}
\OperatorTok{<}\NormalTok{/}\ExtensionTok{body}\OperatorTok{>}
\end{Highlighting}
\end{Shaded}

The content of the page can also be changed using media queries.

\begin{Shaded}
\begin{Highlighting}[]
\OperatorTok{<}\ExtensionTok{meta}\NormalTok{ name=}\StringTok{"viewport"}\NormalTok{ content=}\StringTok{"width=device-width, initial-scale=1.0"}\OperatorTok{>}
\OperatorTok{<}\ExtensionTok{style}\OperatorTok{>}
    \ExtensionTok{@media}\NormalTok{ (min-width: 500px) }\KeywordTok{\{}
        \ExtensionTok{h1}\NormalTok{::before \{}
            \ExtensionTok{content}\NormalTok{: }\StringTok{"Welcome to My Web Page!"}\KeywordTok{;}
        \KeywordTok{\}}
\NormalTok{    \}}

    \ExtensionTok{@media}\NormalTok{ (max-width: 499px) }\KeywordTok{\{}
        \ExtensionTok{h1}\NormalTok{::before \{}
            \ExtensionTok{content}\NormalTok{: }\StringTok{"Welcome!"}\KeywordTok{;}
        \KeywordTok{\}}
\NormalTok{    \}}
\OperatorTok{<}\NormalTok{/}\ExtensionTok{style}\OperatorTok{>}
\OperatorTok{<}\ExtensionTok{body}\OperatorTok{>}
    \OperatorTok{<}\ExtensionTok{h1}\OperatorTok{><}\NormalTok{/h1}\OperatorTok{>}
\OperatorTok{<}\NormalTok{/}\ExtensionTok{body}\OperatorTok{>}
\end{Highlighting}
\end{Shaded}

\subsection{Flexbox}\label{flexbox}

Flexbox styling allows elements to be dynamically arranged to fit the
screen. With a wide enough screen the div elements will all be in one
row, but as the screen shrinks, they will move down into new rows.

\begin{Shaded}
\begin{Highlighting}[]
\OperatorTok{<}\ExtensionTok{meta}\NormalTok{ name=}\StringTok{"viewport"}\NormalTok{ content=}\StringTok{"width=device-width, initial-scale=1.0"}\OperatorTok{>}
\OperatorTok{<}\ExtensionTok{style}\OperatorTok{>}
    \ExtensionTok{.container}\NormalTok{ \{}
        \ExtensionTok{display}\NormalTok{: flex}\KeywordTok{;}
        \ExtensionTok{flex-wrap}\NormalTok{: wrap}\KeywordTok{;}
\NormalTok{    \}}

    \ExtensionTok{.container} \OperatorTok{>}\NormalTok{ div \{}
        \ExtensionTok{background-color}\NormalTok{: springgreen}\KeywordTok{;}
\NormalTok{    \}}
\OperatorTok{<}\NormalTok{/}\ExtensionTok{style}\OperatorTok{>}
\OperatorTok{<}\ExtensionTok{body}\OperatorTok{>}
    \OperatorTok{<}\ExtensionTok{div}\NormalTok{ class=}\StringTok{"container"}\OperatorTok{>}
        \OperatorTok{<}\ExtensionTok{div}\OperatorTok{>}\NormalTok{Some stuff}\OperatorTok{<}\NormalTok{/div}\OperatorTok{>}
        \OperatorTok{<}\ExtensionTok{div}\OperatorTok{>}\NormalTok{Some stuff}\OperatorTok{<}\NormalTok{/div}\OperatorTok{>}
        \OperatorTok{<}\ExtensionTok{div}\OperatorTok{>}\NormalTok{Some stuff}\OperatorTok{<}\NormalTok{/div}\OperatorTok{>}
    \OperatorTok{<}\NormalTok{/}\ExtensionTok{div}\OperatorTok{>}
\OperatorTok{<}\NormalTok{/}\ExtensionTok{body}\OperatorTok{>}
\end{Highlighting}
\end{Shaded}

\subsection{Grid Styling}\label{grid-styling}

In the following example a grid system is being used for the items being
displayed. the grid creates the first two colums as 200px and the last
column is automatically sized to fill the rest of the remaining screen.

\begin{Shaded}
\begin{Highlighting}[]
\OperatorTok{<}\ExtensionTok{meta}\NormalTok{ name=}\StringTok{"viewport"}\NormalTok{ content=}\StringTok{"width=device-width, initial-scale=1.0"}\OperatorTok{>}
\OperatorTok{<}\ExtensionTok{style}\OperatorTok{>}
    \ExtensionTok{.grid}\NormalTok{ \{}
        \ExtensionTok{background-color}\NormalTok{: green}\KeywordTok{;}
        \ExtensionTok{display}\NormalTok{: grid}\KeywordTok{;}
        \ExtensionTok{grid-column-gap}\NormalTok{: 20px}\KeywordTok{;}
        \ExtensionTok{gird-row-gap}\NormalTok{: 10px}\KeywordTok{;}
        \ExtensionTok{grid-template-columns}\NormalTok{: 200px 200px auto}\KeywordTok{;}
\NormalTok{    \}}

    \ExtensionTok{.grid-item}\NormalTok{ \{}
        \ExtensionTok{background-color}\NormalTok{: white}\KeywordTok{;}
\NormalTok{    \}}
\OperatorTok{<}\NormalTok{/}\ExtensionTok{style}\OperatorTok{>}
\OperatorTok{<}\ExtensionTok{body}\OperatorTok{>}
    \OperatorTok{<}\ExtensionTok{div}\NormalTok{ class=}\StringTok{"grid"}\OperatorTok{>}
        \OperatorTok{<}\ExtensionTok{div}\NormalTok{ class=}\StringTok{"grid-item"}\OperatorTok{>1<}\NormalTok{/div}\OperatorTok{>}
        \OperatorTok{<}\ExtensionTok{div}\NormalTok{ class=}\StringTok{"grid-item"}\OperatorTok{>2<}\NormalTok{/div}\OperatorTok{>}
        \OperatorTok{<}\ExtensionTok{div}\NormalTok{ class=}\StringTok{"grid-item"}\OperatorTok{>3<}\NormalTok{/div}\OperatorTok{>}
    \OperatorTok{<}\NormalTok{/}\ExtensionTok{div}\OperatorTok{>}
\OperatorTok{<}\NormalTok{/}\ExtensionTok{body}\OperatorTok{>}
\end{Highlighting}
\end{Shaded}

\chapter{Bootstrap}\label{bootstrap}

\section{Setup}\label{setup}

The bootstrap stylesheet \texttt{\textless{}link\textgreater{}} can be
used directly \texttt{stackpath.com} by including the following
reference in the \texttt{\textless{}head\textgreater{}} before any other
listed stylesheets.

\begin{Shaded}
\begin{Highlighting}[]
\OperatorTok{<}\FunctionTok{link}\NormalTok{ rel=}\StringTok{"stylesheet"}\NormalTok{ href=}\StringTok{"https://stackpath.bootstrapcdn.com/bootstrap/4.3.1/css/bootstrap.min.css"}\NormalTok{ integrity=}\StringTok{"sha384-ggOyR0iXCbMQv3Xipma34MD+dH/1fQ784/j6cY/iJTQUOhcWr7x9JvoRxT2MZw1T"}\NormalTok{ crossorigin=}\StringTok{"anonymous"}\OperatorTok{>}
\end{Highlighting}
\end{Shaded}

Here is an alternative link to the bootstrap CSS file.

\begin{Shaded}
\begin{Highlighting}[]
\OperatorTok{<}\FunctionTok{link}\NormalTok{ rel=}\StringTok{"stylesheet"}\NormalTok{ href=}\StringTok{"https://maxcdn.bootstrapcdn.com/bootstrap/4.0.0/css/bootstrap.min.css"}\NormalTok{ integrity=}\StringTok{"sha384-Gn5384xqQ1aoWXA+}
\end{Highlighting}
\end{Shaded}

The \texttt{charset} and \texttt{viewport} meta tags are often required
for proper bootstrap responsive behaviors, and should be included when
using the bootstrap css.\\
The \texttt{viewport} line is a responsive meta tag that ensures proper
rendering and touch zooming for mobile devices.

\begin{Shaded}
\begin{Highlighting}[]
\OperatorTok{<}\NormalTok{!}\ExtensionTok{--}\NormalTok{ Required meta tags --}\OperatorTok{>}
    \OperatorTok{<}\ExtensionTok{meta}\NormalTok{ charset=}\StringTok{"utf-8"}\OperatorTok{>}
    \OperatorTok{<}\ExtensionTok{meta}\NormalTok{ name=}\StringTok{"viewport"}\NormalTok{ content=}\StringTok{"width=device-width, initial-scale=1, shrink-to-fit=no"}\OperatorTok{>}
\end{Highlighting}
\end{Shaded}

\section{Columns}\label{columns}

Bootstrap styles a page as 12 columns. In the following code, columns
that are 3/12 columns wide are used.

\begin{Shaded}
\begin{Highlighting}[]
\OperatorTok{<}\ExtensionTok{style}\OperatorTok{>}
\OperatorTok{<}\NormalTok{/}\ExtensionTok{style}\OperatorTok{>}
\OperatorTok{<}\ExtensionTok{body}\OperatorTok{>}
    \OperatorTok{<}\ExtensionTok{div}\NormalTok{ class=}\StringTok{"container"}\OperatorTok{>}
        \OperatorTok{<}\ExtensionTok{div}\NormalTok{ class=}\StringTok{"row"}\OperatorTok{>}
            \OperatorTok{<}\ExtensionTok{div}\NormalTok{ class=}\StringTok{"col=3"}\OperatorTok{>}
                \ExtensionTok{This}\NormalTok{ is stuff}
            \OperatorTok{<}\NormalTok{/}\ExtensionTok{div}\OperatorTok{>}
            \OperatorTok{<}\ExtensionTok{div}\NormalTok{ class=}\StringTok{"col=3"}\OperatorTok{>}
                \ExtensionTok{This}\NormalTok{ is stuff}
            \OperatorTok{<}\NormalTok{/}\ExtensionTok{div}\OperatorTok{>}
            \OperatorTok{<}\ExtensionTok{div}\NormalTok{ class=}\StringTok{"col=3"}\OperatorTok{>}
                \ExtensionTok{This}\NormalTok{ is stuff}
            \OperatorTok{<}\NormalTok{/}\ExtensionTok{div}\OperatorTok{>}
        \OperatorTok{<}\NormalTok{/}\ExtensionTok{div}\OperatorTok{>}
    \OperatorTok{<}\NormalTok{/}\ExtensionTok{div}\OperatorTok{>}
\OperatorTok{<}\NormalTok{/}\ExtensionTok{body}\OperatorTok{>}
\end{Highlighting}
\end{Shaded}

Bootstrap can also style elements to take different amounts of the 12
total columns depending on the screen size. What is happening in the
following code is that the columns being listed take 3/12 columns if the
screen is large, as defined by the bootstrap CSS, and they take 6/12
columns if the screen is small.

\begin{Shaded}
\begin{Highlighting}[]
\OperatorTok{<}\ExtensionTok{style}\OperatorTok{>}
    \ExtensionTok{.row} \OperatorTok{>}\NormalTok{ div\{}
        \ExtensionTok{padding}\NormalTok{: 20px}\KeywordTok{;}
        \ExtensionTok{background-color}\NormalTok{: teal}\KeywordTok{;}
        \ExtensionTok{border}\NormalTok{: 2px solid black}\KeywordTok{;}
\NormalTok{    \}}
\OperatorTok{<}\NormalTok{/}\ExtensionTok{style}\OperatorTok{>}
\OperatorTok{<}\ExtensionTok{body}\OperatorTok{>}
    \OperatorTok{<}\ExtensionTok{div}\NormalTok{ class=}\StringTok{"container"}\OperatorTok{>}
        \OperatorTok{<}\ExtensionTok{div}\NormalTok{ class=}\StringTok{"row"}\OperatorTok{>}
            \OperatorTok{<}\ExtensionTok{div}\NormalTok{ class=}\StringTok{"col-lg-3 col-sm-6"}\OperatorTok{>}
                \ExtensionTok{This}\NormalTok{ is a section.}
            \OperatorTok{<}\NormalTok{/}\ExtensionTok{div}\OperatorTok{>}
            \OperatorTok{<}\ExtensionTok{div}\NormalTok{ class=}\StringTok{"col-lg-3 col-sm-6"}\OperatorTok{>}
                \ExtensionTok{This}\NormalTok{ is a section.}
            \OperatorTok{<}\NormalTok{/}\ExtensionTok{div}\OperatorTok{>}
            \OperatorTok{<}\ExtensionTok{div}\NormalTok{ class=}\StringTok{"col-lg-3 col-sm-6"}\OperatorTok{>}
                \ExtensionTok{This}\NormalTok{ is a section.}
            \OperatorTok{<}\NormalTok{/}\ExtensionTok{div}\OperatorTok{>}
        \OperatorTok{<}\NormalTok{/}\ExtensionTok{div}\OperatorTok{>}
    \OperatorTok{<}\NormalTok{/}\ExtensionTok{div}\OperatorTok{>}
\OperatorTok{<}\NormalTok{/}\ExtensionTok{body}\OperatorTok{>}
\end{Highlighting}
\end{Shaded}

\chapter{Sass}\label{sass}

\section{Intro}\label{intro}

Sass can be installed with npm.

\begin{Shaded}
\begin{Highlighting}[]
\ExtensionTok{npm}\NormalTok{ install -g sass}
\end{Highlighting}
\end{Shaded}

Sass is an extension to CSS that adds functionality, including the
addition of variables. Sass converts its code into CSS that can then be
used by the browser. Below is an example of setting a Sass variable
\texttt{\$color}.

\begin{Shaded}
\begin{Highlighting}[]
\VariableTok{$color}\NormalTok{: }\ExtensionTok{blue}\KeywordTok{;}

\ExtensionTok{ul}\NormalTok{ \{}
    \ExtensionTok{font-size}\NormalTok{:14px}\KeywordTok{;}
    \ExtensionTok{color}\NormalTok{: }\VariableTok{$color}\KeywordTok{;}
\NormalTok{\}}
\end{Highlighting}
\end{Shaded}

A Sass file is then converted to CSS by running Sass and specifying the
output.

\begin{Shaded}
\begin{Highlighting}[]
\ExtensionTok{sass}\NormalTok{ variables.scss variables.css}
\end{Highlighting}
\end{Shaded}

CSS files can be automatically recompiled if any changes are detected
using Sass. Here the \texttt{variables.scss} file is being monitored for
changes and recompiled to \texttt{variables.css} whenever changes are
detected.

\begin{Shaded}
\begin{Highlighting}[]
\ExtensionTok{sass}\NormalTok{ --watch variables.scss:variables.css}
\end{Highlighting}
\end{Shaded}

Github pages actually will automatically compile scss files into css
files when a scss file is comitted to a github repository.

\section{Nesting}\label{nesting}

Styles can be applied to divisions or items within other divisions when
using sass. In the below example, the code within a sass file will style
only those paragraphs that are nested within a \texttt{div} blue and
only the \texttt{ul} wihin \texttt{div} as green. Anything outside of a
\texttt{div} will not be styled by this scss code.

\begin{Shaded}
\begin{Highlighting}[]
\ExtensionTok{div}\NormalTok{ \{                }
    \ExtensionTok{font-size}\NormalTok{: 18px}\KeywordTok{;} 
                     
    \ExtensionTok{p}\NormalTok{ \{              }
        \ExtensionTok{color}\NormalTok{: blue}\KeywordTok{;} 
\NormalTok{    \}                }
                     
    \ExtensionTok{ul}\NormalTok{ \{             }
        \ExtensionTok{color}\NormalTok{: green}\KeywordTok{;}
\NormalTok{    \}                }
\NormalTok{\}                    }
\end{Highlighting}
\end{Shaded}

\section{Inheritance}\label{inheritance}

Sass uses inheritance to create generic governing rules that can then be
extended by other elements. This can be useful for similar elements that
share a number of properties but then have a couple of different
properties.\\
In the following example, a \texttt{\%message} group is created with a
number of different styles, then \texttt{.success} extends
\texttt{\%message} and thereby inherits all of the included styles but
then also has a green background.

\begin{Shaded}
\begin{Highlighting}[]
\ExtensionTok{%message}\NormalTok{ \{                  }
    \ExtensionTok{font-family}\NormalTok{: sans-serif}\KeywordTok{;}
    \ExtensionTok{font-size}\NormalTok{: 18px}\KeywordTok{;}        
    \ExtensionTok{font-weight}\NormalTok{: bold}\KeywordTok{;}      
    \ExtensionTok{border}\NormalTok{: 1px solid black}\KeywordTok{;}
    \ExtensionTok{padding}\NormalTok{: 20px}\KeywordTok{;}          
    \ExtensionTok{margin}\NormalTok{: 20px}\KeywordTok{;}           
\NormalTok{\}                           }
                            
\ExtensionTok{.success}\NormalTok{ \{                  }
    \ExtensionTok{@extend}\NormalTok{ %message}\KeywordTok{;}       
    \ExtensionTok{background-color}\NormalTok{: green}\KeywordTok{;}
\NormalTok{\}                           }
\end{Highlighting}
\end{Shaded}

\chapter{Flask}\label{flask}

\section{Intro}\label{intro-1}

There seems to be a nice full flask tutorial
\href{https://blog.miguelgrinberg.com/post/the-flask-mega-tutorial-part-i-hello-world}{here}
where a blog is created.

Installation can be done through anaconda.

\begin{Shaded}
\begin{Highlighting}[]
\ExtensionTok{conda}\NormalTok{ install -c conda-forge flask}
\end{Highlighting}
\end{Shaded}

Flask code is generally stored within a file called
\texttt{application.py}. Below is a general framework of the flask code
that resides in this file.\\
Line 3 is creating a new flask web application.\\
Flask applications are designed around routes. On line 5, what is
happening is that the code is referring to navigation to the \texttt{/}
or home directory. The two lines below this route give the code of what
to do when a user navigates to that home directory.

\begin{Shaded}
\begin{Highlighting}[]
\ExtensionTok{from}\NormalTok{ flask import Flask   }
                          
\ExtensionTok{app}\NormalTok{ = Flask(__name__)     }
                          
\ExtensionTok{@app.route}\NormalTok{(}\StringTok{"/"}\NormalTok{)           }
\ExtensionTok{def}\NormalTok{ index()}\BuiltInTok{:}              
    \BuiltInTok{return} \StringTok{"Hello, world!"}
\end{Highlighting}
\end{Shaded}

Flask then needs to be told how to import it, by setting the
\texttt{FLASK\_APP} environment variable. If the above code is put into
a file called \texttt{application.py}, just export that filename. It may
be useful if \texttt{application.py} is always going to be the main
filename to just add this code to the bashrc.

\begin{Shaded}
\begin{Highlighting}[]
\BuiltInTok{export} \VariableTok{FLASK_APP=}\NormalTok{application.py}
\end{Highlighting}
\end{Shaded}

If flask is run in debug mode, an app will update anytime a change is
made to the underlying code.

\begin{Shaded}
\begin{Highlighting}[]
\BuiltInTok{export} \VariableTok{FLASK_ENV=}\NormalTok{development}
\end{Highlighting}
\end{Shaded}

The web application can then be run by running a flask webserver.

\begin{Shaded}
\begin{Highlighting}[]
\ExtensionTok{flask}\NormalTok{ run}
\end{Highlighting}
\end{Shaded}

\section{Routes}\label{routes}

Instead of just using the main route, new routes can be created that
when navigated to can have different functionality. In the following
example, the route can be accessed at this address:
\texttt{http://127.0.0.1:5000/david}

\begin{Shaded}
\begin{Highlighting}[]
\ExtensionTok{@app.route}\NormalTok{(}\StringTok{"/david"}\NormalTok{)}
\ExtensionTok{def}\NormalTok{ david()}\BuiltInTok{:}
    \BuiltInTok{return} \StringTok{"Hello, David"}
\end{Highlighting}
\end{Shaded}

The route can take the URL addess information dynamically and use it as
a variable. In the following example, the text in the URL is being read
in as a \texttt{string}, and the value is being assigned to the
\texttt{name} variable.

\begin{Shaded}
\begin{Highlighting}[]
\ExtensionTok{@app.route}\NormalTok{(}\StringTok{"/<string:name>"}\NormalTok{)}
\ExtensionTok{def}\NormalTok{ hello(name)}\BuiltInTok{:}
    \ExtensionTok{name}\NormalTok{ = name.capitalize()}
    \BuiltInTok{return} \StringTok{"Hello, \{\}!"}\NormalTok{.format(name)}
\end{Highlighting}
\end{Shaded}

HTML code can also be included in the python code and can be returned
and interpreted as HTML. In the following example the text that is
getting returned and displayed is being styled as header text on the
webpage.

\begin{Shaded}
\begin{Highlighting}[]
\ExtensionTok{@app.route}\NormalTok{(}\StringTok{"/<string:name>"}\NormalTok{)}
\ExtensionTok{def}\NormalTok{ hello(name)}\BuiltInTok{:}
    \ExtensionTok{name}\NormalTok{ = name.capitalize()}
    \BuiltInTok{return} \StringTok{"<h1>Hello, \{\}!</h1>"}\NormalTok{.format(name)}
\end{Highlighting}
\end{Shaded}

\section{Templates}\label{templates}

Instead of embedding HTML within values that get returned by python
code, HTML files themselves can be served by the python code. Flask can
look for HTML files and it will look in a subdirectory of the main
directory called \texttt{templates}. So in the following example if
there is an HTML file called \texttt{index.html} within a subdirectory
called \texttt{templates}, that html file gets served up by the
following code.

\begin{Shaded}
\begin{Highlighting}[]
\ExtensionTok{@app.route}\NormalTok{(}\StringTok{"/"}\NormalTok{)}
\ExtensionTok{def}\NormalTok{ index()}\BuiltInTok{:}
    \BuiltInTok{return}\NormalTok{ render_template(}\StringTok{"index.html"}\NormalTok{)}
\end{Highlighting}
\end{Shaded}

Variables can also be passed from python to the html files that exist
within the \texttt{templates} directory in order to dynamically alter
the HTML content.\\
In the following code, the \texttt{headline} variable is defined in
python and then getting passed to HTML, and it is common that the
varible names are just kept the same, though in the return function the
second \texttt{headline} refers to the python variable and the first to
the html variable just as is python methods.

\begin{Shaded}
\begin{Highlighting}[]
\ExtensionTok{@app.route}\NormalTok{(}\StringTok{"/"}\NormalTok{)}
\ExtensionTok{def}\NormalTok{ index()}\BuiltInTok{:}
    \ExtensionTok{headline}\NormalTok{ = }\StringTok{"Hello thar"}
    \BuiltInTok{return}\NormalTok{ render_template(}\StringTok{"index.html"}\NormalTok{, headline=headline)}
\end{Highlighting}
\end{Shaded}

Now this \texttt{headline} variable can be used in HTML like so. The
language being used is Jinja code. This functionality can be helpful in
using the very same HTML code but allow it to perform differently.

\begin{Shaded}
\begin{Highlighting}[]
\OperatorTok{<}\ExtensionTok{body}\OperatorTok{>}
    \OperatorTok{<}\ExtensionTok{h1}\OperatorTok{>}\NormalTok{\{\{ headline \}\}}\OperatorTok{<}\NormalTok{/}\ExtensionTok{h1}\OperatorTok{>}
\OperatorTok{<}\NormalTok{/}\ExtensionTok{body}\OperatorTok{>}
\end{Highlighting}
\end{Shaded}

\section{Jinja}\label{jinja}

Jinja brings programming logic gates to HTML. The code must be put into
an html file within the \texttt{templates} directory and used by flask
as shown in the above templates section.

\begin{Shaded}
\begin{Highlighting}[]
\OperatorTok{<}\ExtensionTok{body}\OperatorTok{>}
\NormalTok{    \{}\ExtensionTok{%}\NormalTok{ if new_year %\}}
        \OperatorTok{<}\ExtensionTok{h1}\OperatorTok{>}\NormalTok{Happy New Year!}\OperatorTok{<}\NormalTok{/h1}\OperatorTok{>}
\NormalTok{    \{}\ExtensionTok{%}\NormalTok{ else %\}}
        \OperatorTok{<}\ExtensionTok{h1}\OperatorTok{>}\NormalTok{Go back to work}\OperatorTok{<}\NormalTok{/h1}\OperatorTok{>}
\NormalTok{    \{}\ExtensionTok{%}\NormalTok{ endif %\}}
\OperatorTok{<}\NormalTok{/}\ExtensionTok{body}\OperatorTok{>}
\end{Highlighting}
\end{Shaded}

For loops work pretty similar to python as well. In the following code
\texttt{names} is a python list getting passed from
\texttt{application.py}, and the list items are rendered as a
\texttt{ul} in HTML.

\begin{Shaded}
\begin{Highlighting}[]
\OperatorTok{<}\ExtensionTok{ul}\OperatorTok{>}
\NormalTok{    \{}\ExtensionTok{%}\NormalTok{ for name in names %\}}
        \OperatorTok{<}\ExtensionTok{li}\OperatorTok{>}\NormalTok{\{\{ name \}\}}\OperatorTok{<}\NormalTok{/}\ExtensionTok{li}\OperatorTok{>}
\NormalTok{    \{}\ExtensionTok{%}\NormalTok{ endfor %\}}
\OperatorTok{<}\NormalTok{/}\ExtensionTok{ul}\OperatorTok{>}
\end{Highlighting}
\end{Shaded}

Routes can be referred to by the name of a method within them from HTML.
If the following code exists within the \texttt{application.py} file:

\begin{Shaded}
\begin{Highlighting}[]
\ExtensionTok{@app.route}\NormalTok{(}\StringTok{"/more"}\NormalTok{)}
\ExtensionTok{def}\NormalTok{ more()}\BuiltInTok{:}
    \BuiltInTok{return}\NormalTok{ render_template(}\StringTok{"more.html"}\NormalTok{)}
\end{Highlighting}
\end{Shaded}

The above method \texttt{more()} just refers to another html file, and
this html file can be used from html like so, where the jinja function
\texttt{url\_for} finds the route that contains the \texttt{more()}
method, and then uses that URL as the link in the \texttt{href}.

\begin{Shaded}
\begin{Highlighting}[]
\OperatorTok{<}\ExtensionTok{a}\NormalTok{ href=}\StringTok{"\{\{ url_for('more') \}\}"}\OperatorTok{>}\NormalTok{See more...}\OperatorTok{<}\NormalTok{/a}\OperatorTok{>}
\end{Highlighting}
\end{Shaded}

\subsection{Inheritance}\label{inheritance-1}

Inheritance is useful when html code is being repetitively used, as it
allows a general layout to be defined and they reused and slightly
modified.\\
The first part of the example below is how the general layout html can
be setup in a file called \texttt{layout.html}. In the following code a
block is being defined, and the content it uses is whatever is being
passed under the variable name \texttt{heading}.

\begin{Shaded}
\begin{Highlighting}[]
\OperatorTok{<}\ExtensionTok{h1}\OperatorTok{>}\NormalTok{\{% block heading %\}\{}\ExtensionTok{%}\NormalTok{ endblock %\}}\OperatorTok{<}\NormalTok{/}\ExtensionTok{h1}\OperatorTok{>}
\end{Highlighting}
\end{Shaded}

A separate html file can then inherit all of the html content within
\texttt{layout.html}, but then add something unique to the black defined
above.

\begin{Shaded}
\begin{Highlighting}[]
\NormalTok{\{}\ExtensionTok{%extends} \StringTok{"layout.html"}\NormalTok{ %\}}

\NormalTok{\{}\ExtensionTok{%}\NormalTok{ block heading %\}}
    \ExtensionTok{This}\NormalTok{ is the header}
\NormalTok{\{}\ExtensionTok{%}\NormalTok{ endblock %\}}
\end{Highlighting}
\end{Shaded}

\subsection{Forms}\label{forms-1}

Forms or fields allow input to be captures from the user and then used
to perform other functions. In the below example, the \texttt{action} is
being set as the route that contains teh \texttt{hello} method, and the
manner in which the information is sent is \texttt{post}. The
\texttt{post} method is different from other methods like the
\texttt{get} method where information is submitted and a result is
returned; may need to read more on this. The \texttt{get} method will
place the text from the form into the URL. The information being input
into the \texttt{\textless{}input\textgreater{}} field is named
\texttt{"name"} and this varible can be then passed to
\texttt{Application.py}.

\begin{Shaded}
\begin{Highlighting}[]
\OperatorTok{<}\ExtensionTok{form}\NormalTok{ action=}\StringTok{"\{\{ url_for('hello') \}\}"}\NormalTok{ method=}\StringTok{"post"}\OperatorTok{>}
    \OperatorTok{<}\ExtensionTok{input}\NormalTok{ type=}\StringTok{"text"}\NormalTok{ name=}\StringTok{"name"}\NormalTok{ placeholder=}\StringTok{"Enter Your Name"}\NormalTok{\}}
    \OperatorTok{<}\ExtensionTok{button}\OperatorTok{>}\NormalTok{Submit}\OperatorTok{<}\NormalTok{/button}\OperatorTok{>}
\OperatorTok{<}\NormalTok{/}\ExtensionTok{form}\OperatorTok{>}
\end{Highlighting}
\end{Shaded}

Within \texttt{Application.py} the text that was input into the form is
then used to send to \texttt{hello.html}. here the \texttt{name}
variable gets set by using the \texttt{form.get} method to retrieve the
information, and then the \texttt{hello.html} template is rendered by
passing the \texttt{name} variable tot by using the \texttt{form.get}
method to retrieve the information, and then the \texttt{hello.html}
template is rendered by passing the \texttt{name} variable to it.

\begin{Shaded}
\begin{Highlighting}[]
\ExtensionTok{from}\NormalTok{ flask import Flask, render_template, request}

\ExtensionTok{@app.route}\NormalTok{(}\StringTok{"/hello"}\NormalTok{, methods=[}\StringTok{"POST"}\NormalTok{])             }
\ExtensionTok{def}\NormalTok{ hello()}\BuiltInTok{:}                                       
    \ExtensionTok{name}\NormalTok{ = request.form.get(}\StringTok{"name"}\NormalTok{)                }
    \BuiltInTok{return}\NormalTok{ render_template(}\StringTok{"hello.html"}\NormalTok{, name=name)}
\end{Highlighting}
\end{Shaded}

\texttt{hello.html} can then use the \texttt{name} variable to display
the text entered into the form.

\begin{Shaded}
\begin{Highlighting}[]
\ExtensionTok{Hello}\NormalTok{, \{\{ name \}\}!}
\end{Highlighting}
\end{Shaded}

\subsection{Sessions}\label{sessions}

Sessions allow data to be stored, and as long as the webserver is still
running (or the data can be saved), and furthermore allows each user to
have data that is specific to their individual session. First
\texttt{flask\_session} needs to be installed.

\begin{Shaded}
\begin{Highlighting}[]
\ExtensionTok{pip}\NormalTok{ install flask-session}
\end{Highlighting}
\end{Shaded}

Here in \texttt{application.py} a \texttt{notes} list is being created
to store data that is being input. The route uses both the \texttt{get}
and \texttt{post} methods to get data from the user. And the form sends
the \texttt{note} variable.\\
The \texttt{notes} list is created using \texttt{session{[}"notes"{]}}
as this will create a list that is stored in a cookie and is specific to
the user.

\begin{Shaded}
\begin{Highlighting}[]
\ExtensionTok{from}\NormalTok{ flask import Flask, render_template, request, session}
\ExtensionTok{from}\NormalTok{ flask_session import Session                         }
                                                          
\ExtensionTok{app}\NormalTok{ = Flask(__name__)                                     }
                                                          
\ExtensionTok{app.config}\NormalTok{[}\StringTok{"SESSION_PERMANENT"}\NormalTok{] = False                   }
\ExtensionTok{app.config}\NormalTok{[}\StringTok{"SESSION_TYPE"}\NormalTok{] = }\StringTok{"filesystem"}                 
\ExtensionTok{Session}\NormalTok{(app)                                              }
                                                          
\ExtensionTok{@app.route}\NormalTok{(}\StringTok{"/"}\NormalTok{, methods=[}\StringTok{"GET"}\NormalTok{, }\StringTok{"POST"}\NormalTok{])                  }
\ExtensionTok{def}\NormalTok{ index()}\BuiltInTok{:}                                              
    \KeywordTok{if} \ExtensionTok{session.get}\NormalTok{(}\StringTok{"notes"}\NormalTok{) }\ExtensionTok{is}\NormalTok{ None:}
        \ExtensionTok{session}\NormalTok{[}\StringTok{"notes"}\NormalTok{] = []       }

    \KeywordTok{if} \ExtensionTok{request.method}\NormalTok{ == }\StringTok{"POST"}\NormalTok{:                          }
        \ExtensionTok{note}\NormalTok{ = request.form.get(}\StringTok{"note"}\NormalTok{)                   }
        \ExtensionTok{notes.append}\NormalTok{(note)                                }
                                                          
    \BuiltInTok{return}\NormalTok{ render_template(}\StringTok{"index.html"}\NormalTok{, notes=notes)     }
\end{Highlighting}
\end{Shaded}

Then in \texttt{index.html} the form is created to send the data to the
\texttt{application.py} file. And since \texttt{notes} is an ever
growing list, the list is rendered in the html.

\begin{Shaded}
\begin{Highlighting}[]
\OperatorTok{<}\ExtensionTok{ul}\OperatorTok{>}                                                             
\NormalTok{    \{}\ExtensionTok{%}\NormalTok{ for note in notes %\}                                      }
        \OperatorTok{<}\ExtensionTok{li}\OperatorTok{>}\NormalTok{\{\{ note \}\}}\OperatorTok{<}\NormalTok{/}\ExtensionTok{li}\OperatorTok{>}                                      
\NormalTok{    \{}\ExtensionTok{%}\NormalTok{ endfor %\}                                                 }
\OperatorTok{<}\NormalTok{/}\ExtensionTok{ul}\OperatorTok{>}                                                            
                                                                 
\OperatorTok{<}\ExtensionTok{form}\NormalTok{ action=}\StringTok{"\{\{ url_for('index') \}\}"}\NormalTok{ method=}\StringTok{"post"}\OperatorTok{>}             
    \OperatorTok{<}\ExtensionTok{input}\NormalTok{ type=}\StringTok{"text"}\NormalTok{ name=}\StringTok{"note"}\NormalTok{ placeholder=}\StringTok{"Enter Note Here"}\OperatorTok{>}
    \OperatorTok{<}\ExtensionTok{button}\OperatorTok{>}\NormalTok{Add Note}\OperatorTok{<}\NormalTok{/button}\OperatorTok{>}                                    
\OperatorTok{<}\NormalTok{/}\ExtensionTok{form}\OperatorTok{>}                                                          
\end{Highlighting}
\end{Shaded}

\chapter{SQL}\label{sql}

\section{Intro}\label{intro-2}

SQL stands for structured query language that is designed to facilitate
accessing data that is structured into table form. PostgreSQL is a
version of SQL that is used in these notes.

\section{Installation}\label{installation}

The method I can easily get to work is by installing through
\texttt{apt}.

\begin{Shaded}
\begin{Highlighting}[]
\FunctionTok{sudo}\NormalTok{ apt-get install postgresql postgresql-client}
\end{Highlighting}
\end{Shaded}

Both the default database user and default database are called postgres,
so switch to that user.

\begin{Shaded}
\begin{Highlighting}[]
\FunctionTok{sudo}\NormalTok{ -u postgres bash}
\end{Highlighting}
\end{Shaded}

Then start a server.

\begin{Shaded}
\begin{Highlighting}[]
\ExtensionTok{psql}
\end{Highlighting}
\end{Shaded}

I have not been able to get the installation working through a conda
install but here is the reference to postgresql.

\begin{Shaded}
\begin{Highlighting}[]
\ExtensionTok{conda}\NormalTok{ install -c conda-forge postgresql}
\end{Highlighting}
\end{Shaded}

\section{Creating a Database}\label{creating-a-database}

Below is the general syntax to create a PostgreSQL database.\\
The \texttt{id} is being used in a manner similar to an index that will
just number each of the items.\\
The next three lines are all different data, and are of types
\texttt{VARCHAR} and \texttt{INTEGER}.\\
The \texttt{NOT\ NULL} aspect will cause the server to reject the entry
if some data is added to the database but that value is not included.

\begin{Shaded}
\begin{Highlighting}[]
\ExtensionTok{CREATE}\NormalTok{ TABLE flights (           }
    \FunctionTok{id}\NormalTok{ SERIAL PRIMARY KEY,       }
    \ExtensionTok{origin}\NormalTok{ VARCHAR NOT NULL,     }
    \ExtensionTok{destination}\NormalTok{ VARCHAR NOT NULL,}
    \ExtensionTok{duration}\NormalTok{ INTEGER NOT NULL    }
\NormalTok{);                               }
\end{Highlighting}
\end{Shaded}

Display the currently created databases.

\begin{Shaded}
\begin{Highlighting}[]
\NormalTok{\textbackslash{}}\ExtensionTok{d}
\end{Highlighting}
\end{Shaded}

Insert data into the flights database.

\begin{Shaded}
\begin{Highlighting}[]
\ExtensionTok{INSERT}\NormalTok{ INTO flights (origin, destination, duration) }\ExtensionTok{VALUES}\NormalTok{ (}\StringTok{'New York'}\NormalTok{, }\StringTok{'London'}\NormalTok{, 415);}
\end{Highlighting}
\end{Shaded}

Select all the data from flights.

\begin{Shaded}
\begin{Highlighting}[]
\ExtensionTok{SELECT}\NormalTok{ * FROM flights}\KeywordTok{;}
\end{Highlighting}
\end{Shaded}

Select only the origin and destination columns from flights.

\begin{Shaded}
\begin{Highlighting}[]
\ExtensionTok{SELECT}\NormalTok{ origin, destination FROM flights}\KeywordTok{;}
\end{Highlighting}
\end{Shaded}

Select only the data in flights where the id is 3.

\begin{Shaded}
\begin{Highlighting}[]
\ExtensionTok{SELECT}\NormalTok{ * FROM flights WHERE id = 3}\KeywordTok{;}
\end{Highlighting}
\end{Shaded}

Select only the data in flights that have an origin of New York.

\begin{Shaded}
\begin{Highlighting}[]
\ExtensionTok{SELECT}\NormalTok{ * FROM flights WHERE origin = }\StringTok{'New York'}\KeywordTok{;}
\end{Highlighting}
\end{Shaded}

Boolean logic data selection from flights.

\begin{Shaded}
\begin{Highlighting}[]
\ExtensionTok{SELECT}\NormalTok{ * FROM flights WHERE destination = }\StringTok{'Paris'}\NormalTok{ AND duration }\OperatorTok{>}\NormalTok{ 500}\KeywordTok{;}
\end{Highlighting}
\end{Shaded}

\chapter{Hackernews}\label{hackernews}

\section{Setup}\label{setup-1}

First make sure create react app is installed. The project here follows
\href{https://www.youtube.com/watch?v=oGB_VPrld0U\&list=PLTTC1K14KAxHj6AftnRUD28SQaoVauvl3}{this}
tutorial. There are lots of other good looking tutorials like
\href{https://www.freecodecamp.org/news/the-react-handbook-b71c27b0a795/}{The
React Handbook}, and others at
\href{https://gitconnected.com/learn/react}{gitconnected}.

\begin{Shaded}
\begin{Highlighting}[]
\ExtensionTok{npm}\NormalTok{ i -g create-react-app}
\end{Highlighting}
\end{Shaded}

Then create a new directory for the app.

\begin{Shaded}
\begin{Highlighting}[]
\ExtensionTok{create-react-app}\NormalTok{ hacker-news-clone}
\end{Highlighting}
\end{Shaded}

Change into the newly created directory and then create a file to handle
environmental variables.

\begin{Shaded}
\begin{Highlighting}[]
\BuiltInTok{cd}\NormalTok{ hacker-news-clone}
\FunctionTok{touch}\NormalTok{ .env}
\end{Highlighting}
\end{Shaded}

Within the \texttt{.env} file refer to the \texttt{src} folder. This
will allow dependencies to be more easily imported. Add the following to
the \texttt{.env} file.

\begin{Shaded}
\begin{Highlighting}[]
\VariableTok{NODE_PATH=}\NormalTok{src}
\end{Highlighting}
\end{Shaded}

Make a components directory within \texttt{src} to hold all of the
components for the project.

\begin{Shaded}
\begin{Highlighting}[]
\FunctionTok{mkdir}\NormalTok{ -p src/components/App}
\end{Highlighting}
\end{Shaded}

Make a services directory within \texttt{src} to add additional
functionality to the app and reference other site APIs.

\begin{Shaded}
\begin{Highlighting}[]
\FunctionTok{mkdir}\NormalTok{ src/services}
\end{Highlighting}
\end{Shaded}

Make a styles directory within \texttt{src} to add styles that can be
used across the app.

\begin{Shaded}
\begin{Highlighting}[]
\FunctionTok{mkdir}\NormalTok{ -p src/styles}
\end{Highlighting}
\end{Shaded}

Make a store directory within \texttt{src} to add styles that will add
Redux function.

\begin{Shaded}
\begin{Highlighting}[]
\FunctionTok{mkdir}\NormalTok{ -p src/store}
\end{Highlighting}
\end{Shaded}

Make a utils directory within \texttt{src} for shared functions across
the app.

\begin{Shaded}
\begin{Highlighting}[]
\FunctionTok{mkdir}\NormalTok{ -p src/utils}
\end{Highlighting}
\end{Shaded}

Now move \texttt{App.js} to components just to keep the components
bundled together. Rename \texttt{App.js} to index so that it can be
imported from the mycomponents app.

\begin{Shaded}
\begin{Highlighting}[]
\FunctionTok{mv}\NormalTok{ src/App*js src/components/App/}
\FunctionTok{mv}\NormalTok{ src/components/App/App.js src/components/App/index.js}
\FunctionTok{mv}\NormalTok{ src/logo.svg src/components/App/}
\end{Highlighting}
\end{Shaded}

Delete the css files because style components will be used instead.

\begin{Shaded}
\begin{Highlighting}[]
\FunctionTok{rm}\NormalTok{ src/*css}
\end{Highlighting}
\end{Shaded}

Remove the imports of the css files in
\texttt{src/components/App/index.js}.

\begin{Shaded}
\begin{Highlighting}[]
\ExtensionTok{import} \StringTok{'./App.css'}\KeywordTok{;}
\end{Highlighting}
\end{Shaded}

And remove the import within \texttt{src/index.js}.

\begin{Shaded}
\begin{Highlighting}[]
\ExtensionTok{import} \StringTok{'./index.css'}\KeywordTok{;}
\end{Highlighting}
\end{Shaded}

Now create some styles to be used throughout the app.

\begin{Shaded}
\begin{Highlighting}[]
\FunctionTok{mkdir}\NormalTok{ src/styles}
\FunctionTok{touch}\NormalTok{ src/styles/globals.js}
\FunctionTok{touch}\NormalTok{ src/styles/palette.js}
\end{Highlighting}
\end{Shaded}

The js files above contain routine code that can be copied from the
author's \href{https://github.com/treyhuffine/hn-clone}{github page}.
Alternatively, here is the code for \texttt{global.js}.

\begin{Shaded}
\begin{Highlighting}[]

\ExtensionTok{import}\NormalTok{ \{ injectGlobal \} }\ExtensionTok{from} \StringTok{'styled-components'}\KeywordTok{;}
\ExtensionTok{import}\NormalTok{ \{ colorsDark \} }\ExtensionTok{from} \StringTok{'./palette'}\KeywordTok{;}

\ExtensionTok{const}\NormalTok{ setGlobalStyles = () =}\OperatorTok{>}
  \ExtensionTok{injectGlobal}\KeywordTok{`}
    \ExtensionTok{*}\NormalTok{ \{}
      \ExtensionTok{box-sizing}\NormalTok{: border-box}\KeywordTok{;}
\NormalTok{    \}}
    \ExtensionTok{html}\NormalTok{, body \{}
      \ExtensionTok{font-family}\NormalTok{: Lato,Helvetica-Neue,Helvetica,Arial,sans-serif}\KeywordTok{;}
      \ExtensionTok{width}\NormalTok{: 100vw}\KeywordTok{;}
      \ExtensionTok{overflow-x}\NormalTok{: hidden}\KeywordTok{;}
      \ExtensionTok{margin}\NormalTok{: 0}\KeywordTok{;}
      \ExtensionTok{padding}\NormalTok{: 0}\KeywordTok{;}
      \ExtensionTok{min-height}\NormalTok{: 100vh}\KeywordTok{;}
      \ExtensionTok{background-color}\NormalTok{: }\VariableTok{$\{colorsDark}\ErrorTok{.background}\VariableTok{\}}\KeywordTok{;}
\NormalTok{    \}}
    \ExtensionTok{ul}\NormalTok{ \{}
      \ExtensionTok{list-style}\NormalTok{: none}\KeywordTok{;}
      \ExtensionTok{padding}\NormalTok{: 0}\KeywordTok{;}
\NormalTok{    \}}
    \ExtensionTok{a}\NormalTok{ \{}
      \ExtensionTok{text-decoration}\NormalTok{: none}\KeywordTok{;}
      \KeywordTok{&}\NormalTok{:}\ExtensionTok{visited}\NormalTok{ \{}
        \ExtensionTok{color}\NormalTok{: inherit}\KeywordTok{;}
\NormalTok{      \}}
\NormalTok{    \}}
  \KeywordTok{`;}

\BuiltInTok{export} \VariableTok{default} \VariableTok{setGlobalStyles}\NormalTok{;}
\end{Highlighting}
\end{Shaded}

And here is the code for \texttt{palette.js}.

\begin{Shaded}
\begin{Highlighting}[]
\BuiltInTok{export} \VariableTok{const} \VariableTok{colorsDark} \VariableTok{=}\NormalTok{ \{}
  \ExtensionTok{background}\NormalTok{: }\StringTok{'#272727'}\NormalTok{,}
  \ExtensionTok{backgroundSecondary}\NormalTok{: }\StringTok{'#393C3E'}\NormalTok{,}
  \ExtensionTok{text}\NormalTok{: }\StringTok{'#bfbebe'}\NormalTok{,}
  \ExtensionTok{textSecondary}\NormalTok{: }\StringTok{'#848886'}\NormalTok{,}
  \ExtensionTok{border}\NormalTok{: }\StringTok{'#272727'}\NormalTok{,}
\NormalTok{\};}

\BuiltInTok{export} \VariableTok{const} \VariableTok{colorsLight} \VariableTok{=}\NormalTok{ \{}
  \ExtensionTok{background}\NormalTok{: }\StringTok{'#EAEAEA'}\NormalTok{,}
  \ExtensionTok{backgroundSecondary}\NormalTok{: }\StringTok{'#F8F8F8'}\NormalTok{,}
  \ExtensionTok{text}\NormalTok{: }\StringTok{'#848886'}\NormalTok{,}
  \ExtensionTok{textSecondary}\NormalTok{: }\StringTok{'#aaaaaa'}\NormalTok{,}
  \ExtensionTok{border}\NormalTok{: }\StringTok{'#EAEAEA'}\NormalTok{,}
\NormalTok{\};}
\end{Highlighting}
\end{Shaded}

\bibliography{book.bib,packages.bib}


\end{document}
