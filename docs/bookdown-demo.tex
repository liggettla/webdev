\documentclass[]{book}
\usepackage{lmodern}
\usepackage{amssymb,amsmath}
\usepackage{ifxetex,ifluatex}
\usepackage{fixltx2e} % provides \textsubscript
\ifnum 0\ifxetex 1\fi\ifluatex 1\fi=0 % if pdftex
  \usepackage[T1]{fontenc}
  \usepackage[utf8]{inputenc}
\else % if luatex or xelatex
  \ifxetex
    \usepackage{mathspec}
  \else
    \usepackage{fontspec}
  \fi
  \defaultfontfeatures{Ligatures=TeX,Scale=MatchLowercase}
\fi
% use upquote if available, for straight quotes in verbatim environments
\IfFileExists{upquote.sty}{\usepackage{upquote}}{}
% use microtype if available
\IfFileExists{microtype.sty}{%
\usepackage{microtype}
\UseMicrotypeSet[protrusion]{basicmath} % disable protrusion for tt fonts
}{}
\usepackage[margin=1in]{geometry}
\usepackage{hyperref}
\hypersetup{unicode=true,
            pdftitle={Bootstrap and React for Web Development},
            pdfauthor={L A Liggett},
            pdfborder={0 0 0},
            breaklinks=true}
\urlstyle{same}  % don't use monospace font for urls
\usepackage{natbib}
\bibliographystyle{apalike}
\usepackage{color}
\usepackage{fancyvrb}
\newcommand{\VerbBar}{|}
\newcommand{\VERB}{\Verb[commandchars=\\\{\}]}
\DefineVerbatimEnvironment{Highlighting}{Verbatim}{commandchars=\\\{\}}
% Add ',fontsize=\small' for more characters per line
\usepackage{framed}
\definecolor{shadecolor}{RGB}{248,248,248}
\newenvironment{Shaded}{\begin{snugshade}}{\end{snugshade}}
\newcommand{\KeywordTok}[1]{\textcolor[rgb]{0.13,0.29,0.53}{\textbf{#1}}}
\newcommand{\DataTypeTok}[1]{\textcolor[rgb]{0.13,0.29,0.53}{#1}}
\newcommand{\DecValTok}[1]{\textcolor[rgb]{0.00,0.00,0.81}{#1}}
\newcommand{\BaseNTok}[1]{\textcolor[rgb]{0.00,0.00,0.81}{#1}}
\newcommand{\FloatTok}[1]{\textcolor[rgb]{0.00,0.00,0.81}{#1}}
\newcommand{\ConstantTok}[1]{\textcolor[rgb]{0.00,0.00,0.00}{#1}}
\newcommand{\CharTok}[1]{\textcolor[rgb]{0.31,0.60,0.02}{#1}}
\newcommand{\SpecialCharTok}[1]{\textcolor[rgb]{0.00,0.00,0.00}{#1}}
\newcommand{\StringTok}[1]{\textcolor[rgb]{0.31,0.60,0.02}{#1}}
\newcommand{\VerbatimStringTok}[1]{\textcolor[rgb]{0.31,0.60,0.02}{#1}}
\newcommand{\SpecialStringTok}[1]{\textcolor[rgb]{0.31,0.60,0.02}{#1}}
\newcommand{\ImportTok}[1]{#1}
\newcommand{\CommentTok}[1]{\textcolor[rgb]{0.56,0.35,0.01}{\textit{#1}}}
\newcommand{\DocumentationTok}[1]{\textcolor[rgb]{0.56,0.35,0.01}{\textbf{\textit{#1}}}}
\newcommand{\AnnotationTok}[1]{\textcolor[rgb]{0.56,0.35,0.01}{\textbf{\textit{#1}}}}
\newcommand{\CommentVarTok}[1]{\textcolor[rgb]{0.56,0.35,0.01}{\textbf{\textit{#1}}}}
\newcommand{\OtherTok}[1]{\textcolor[rgb]{0.56,0.35,0.01}{#1}}
\newcommand{\FunctionTok}[1]{\textcolor[rgb]{0.00,0.00,0.00}{#1}}
\newcommand{\VariableTok}[1]{\textcolor[rgb]{0.00,0.00,0.00}{#1}}
\newcommand{\ControlFlowTok}[1]{\textcolor[rgb]{0.13,0.29,0.53}{\textbf{#1}}}
\newcommand{\OperatorTok}[1]{\textcolor[rgb]{0.81,0.36,0.00}{\textbf{#1}}}
\newcommand{\BuiltInTok}[1]{#1}
\newcommand{\ExtensionTok}[1]{#1}
\newcommand{\PreprocessorTok}[1]{\textcolor[rgb]{0.56,0.35,0.01}{\textit{#1}}}
\newcommand{\AttributeTok}[1]{\textcolor[rgb]{0.77,0.63,0.00}{#1}}
\newcommand{\RegionMarkerTok}[1]{#1}
\newcommand{\InformationTok}[1]{\textcolor[rgb]{0.56,0.35,0.01}{\textbf{\textit{#1}}}}
\newcommand{\WarningTok}[1]{\textcolor[rgb]{0.56,0.35,0.01}{\textbf{\textit{#1}}}}
\newcommand{\AlertTok}[1]{\textcolor[rgb]{0.94,0.16,0.16}{#1}}
\newcommand{\ErrorTok}[1]{\textcolor[rgb]{0.64,0.00,0.00}{\textbf{#1}}}
\newcommand{\NormalTok}[1]{#1}
\usepackage{longtable,booktabs}
\usepackage{graphicx,grffile}
\makeatletter
\def\maxwidth{\ifdim\Gin@nat@width>\linewidth\linewidth\else\Gin@nat@width\fi}
\def\maxheight{\ifdim\Gin@nat@height>\textheight\textheight\else\Gin@nat@height\fi}
\makeatother
% Scale images if necessary, so that they will not overflow the page
% margins by default, and it is still possible to overwrite the defaults
% using explicit options in \includegraphics[width, height, ...]{}
\setkeys{Gin}{width=\maxwidth,height=\maxheight,keepaspectratio}
\IfFileExists{parskip.sty}{%
\usepackage{parskip}
}{% else
\setlength{\parindent}{0pt}
\setlength{\parskip}{6pt plus 2pt minus 1pt}
}
\setlength{\emergencystretch}{3em}  % prevent overfull lines
\providecommand{\tightlist}{%
  \setlength{\itemsep}{0pt}\setlength{\parskip}{0pt}}
\setcounter{secnumdepth}{5}
% Redefines (sub)paragraphs to behave more like sections
\ifx\paragraph\undefined\else
\let\oldparagraph\paragraph
\renewcommand{\paragraph}[1]{\oldparagraph{#1}\mbox{}}
\fi
\ifx\subparagraph\undefined\else
\let\oldsubparagraph\subparagraph
\renewcommand{\subparagraph}[1]{\oldsubparagraph{#1}\mbox{}}
\fi

%%% Use protect on footnotes to avoid problems with footnotes in titles
\let\rmarkdownfootnote\footnote%
\def\footnote{\protect\rmarkdownfootnote}

%%% Change title format to be more compact
\usepackage{titling}

% Create subtitle command for use in maketitle
\providecommand{\subtitle}[1]{
  \posttitle{
    \begin{center}\large#1\end{center}
    }
}

\setlength{\droptitle}{-2em}

  \title{Bootstrap and React for Web Development}
    \pretitle{\vspace{\droptitle}\centering\huge}
  \posttitle{\par}
    \author{L A Liggett}
    \preauthor{\centering\large\emph}
  \postauthor{\par}
      \predate{\centering\large\emph}
  \postdate{\par}
    \date{2019-07-30}

\usepackage{booktabs}
\usepackage{amsthm}
\makeatletter
\def\thm@space@setup{%
  \thm@preskip=8pt plus 2pt minus 4pt
  \thm@postskip=\thm@preskip
}
\makeatother

\begin{document}
\maketitle

{
\setcounter{tocdepth}{1}
\tableofcontents
}
\chapter{Introduction}\label{introduction}

\chapter{HTML}\label{html}

\section{HTML Properties}\label{html-properties}

The head tag allows metadata to be labeled, the text of title for
instance is typically listed in the tab or the status bar of the page in
a browser.

\begin{Shaded}
\begin{Highlighting}[]
\OperatorTok{<}\NormalTok{!}\ExtensionTok{DOCTYPE}\NormalTok{ html}\OperatorTok{>}         
\OperatorTok{<}\ExtensionTok{html}\OperatorTok{>}                  
    \OperatorTok{<}\FunctionTok{head}\OperatorTok{>}              
        \OperatorTok{<}\ExtensionTok{title}\OperatorTok{>}         
            \ExtensionTok{My}\NormalTok{ Web Page!}
        \OperatorTok{<}\NormalTok{/}\ExtensionTok{title}\OperatorTok{>}        
    \OperatorTok{<}\NormalTok{/}\ExtensionTok{head}\OperatorTok{>}             
\OperatorTok{<}\NormalTok{/}\ExtensionTok{html}\OperatorTok{>}                 
\end{Highlighting}
\end{Shaded}

The body specifies text for the page body.

\begin{Shaded}
\begin{Highlighting}[]
\OperatorTok{<}\NormalTok{!}\ExtensionTok{DOCTYPE}\NormalTok{ html}\OperatorTok{>}         
\OperatorTok{<}\ExtensionTok{html}\OperatorTok{>}                  
    \OperatorTok{<}\ExtensionTok{body}\OperatorTok{>}              
        \ExtensionTok{Hello}\NormalTok{, world!   }
    \OperatorTok{<}\NormalTok{/}\ExtensionTok{body}\OperatorTok{>}             
\OperatorTok{<}\NormalTok{/}\ExtensionTok{html}\OperatorTok{>}                 
\end{Highlighting}
\end{Shaded}

Headings specifies header text of increasingly small sizes.

\begin{Shaded}
\begin{Highlighting}[]
\OperatorTok{<}\NormalTok{!}\ExtensionTok{DOCTYPE}\NormalTok{ html}\OperatorTok{>}         
\OperatorTok{<}\ExtensionTok{html}\OperatorTok{>}                  
    \OperatorTok{<}\ExtensionTok{body}\OperatorTok{>}              
        \OperatorTok{<}\ExtensionTok{h1}\OperatorTok{>}\NormalTok{This is the largest headline}\OperatorTok{<}\NormalTok{/h1}\OperatorTok{>}        
        \OperatorTok{<}\ExtensionTok{h2}\OperatorTok{>}\NormalTok{This is also a large headline}\OperatorTok{<}\NormalTok{/h2}\OperatorTok{>}       
        \OperatorTok{<}\ExtensionTok{h3}\OperatorTok{>}\NormalTok{This is a slightly smaller headline}\OperatorTok{<}\NormalTok{/h3}\OperatorTok{>} 
        \OperatorTok{<}\ExtensionTok{h4}\OperatorTok{>}\NormalTok{This is an even smaller headline}\OperatorTok{<}\NormalTok{/h4}\OperatorTok{>}    
        \OperatorTok{<}\ExtensionTok{h5}\OperatorTok{>}\NormalTok{This is the second-smallest headline}\OperatorTok{<}\NormalTok{/h5}\OperatorTok{>}
        \OperatorTok{<}\ExtensionTok{h6}\OperatorTok{>}\NormalTok{This is the smallest headline}\OperatorTok{<}\NormalTok{/h6}\OperatorTok{>}       
    \OperatorTok{<}\NormalTok{/}\ExtensionTok{body}\OperatorTok{>}             
\OperatorTok{<}\NormalTok{/}\ExtensionTok{html}\OperatorTok{>}                 
\end{Highlighting}
\end{Shaded}

Unordered lists specify bullet points.

\begin{Shaded}
\begin{Highlighting}[]
\OperatorTok{<}\NormalTok{!}\ExtensionTok{DOCTYPE}\NormalTok{ html}\OperatorTok{>}         
\OperatorTok{<}\ExtensionTok{html}\OperatorTok{>}                  
    \OperatorTok{<}\ExtensionTok{body}\OperatorTok{>}              
        \ExtensionTok{An}\NormalTok{ Unordered List:           }
        \OperatorTok{<}\ExtensionTok{ul}\OperatorTok{>}                         
            \OperatorTok{<}\ExtensionTok{li}\OperatorTok{>}\NormalTok{One Item}\OperatorTok{<}\NormalTok{/li}\OperatorTok{>}        
            \OperatorTok{<}\ExtensionTok{li}\OperatorTok{>}\NormalTok{Another Item}\OperatorTok{<}\NormalTok{/li}\OperatorTok{>}    
            \OperatorTok{<}\ExtensionTok{li}\OperatorTok{>}\NormalTok{Yet Another Item}\OperatorTok{<}\NormalTok{/li}\OperatorTok{>}
        \OperatorTok{<}\NormalTok{/}\ExtensionTok{ul}\OperatorTok{>}                        
    \OperatorTok{<}\NormalTok{/}\ExtensionTok{body}\OperatorTok{>}             
\OperatorTok{<}\NormalTok{/}\ExtensionTok{html}\OperatorTok{>}                 
\end{Highlighting}
\end{Shaded}

Ordered lists number lines in increasing order.

\begin{Shaded}
\begin{Highlighting}[]
\OperatorTok{<}\NormalTok{!}\ExtensionTok{DOCTYPE}\NormalTok{ html}\OperatorTok{>}         
\OperatorTok{<}\ExtensionTok{html}\OperatorTok{>}                  
    \OperatorTok{<}\ExtensionTok{body}\OperatorTok{>}              
        \ExtensionTok{An}\NormalTok{ Ordered List:}
        \OperatorTok{<}\ExtensionTok{ol}\OperatorTok{>}                          
            \OperatorTok{<}\ExtensionTok{li}\OperatorTok{>}\NormalTok{First Item}\OperatorTok{<}\NormalTok{/li}\OperatorTok{>}       
            \OperatorTok{<}\ExtensionTok{li}\OperatorTok{>}\NormalTok{Second Item}\OperatorTok{<}\NormalTok{/li}\OperatorTok{>}      
            \OperatorTok{<}\ExtensionTok{li}\OperatorTok{>}\NormalTok{Another Item Here}\OperatorTok{<}\NormalTok{/li}\OperatorTok{>}
            \OperatorTok{<}\ExtensionTok{li}\OperatorTok{>}\NormalTok{Fourth Item}\OperatorTok{<}\NormalTok{/li}\OperatorTok{>}      
        \OperatorTok{<}\NormalTok{/}\ExtensionTok{ol}\OperatorTok{>}                         
    \OperatorTok{<}\NormalTok{/}\ExtensionTok{body}\OperatorTok{>}             
\OperatorTok{<}\NormalTok{/}\ExtensionTok{html}\OperatorTok{>}                 
\end{Highlighting}
\end{Shaded}

The image tag refers to and inserts an image as an html attribute. The
alt gives alternative code if the image is missing. The height and width
sets the image size in number of pixels. When the image size is set to
50\% sets the image size dynamically to 50\% of the browser width or
height.

\begin{Shaded}
\begin{Highlighting}[]
\OperatorTok{<}\NormalTok{!}\ExtensionTok{DOCTYPE}\NormalTok{ html}\OperatorTok{>}         
\OperatorTok{<}\ExtensionTok{html}\OperatorTok{>}                  
    \OperatorTok{<}\ExtensionTok{body}\OperatorTok{>}              
        \OperatorTok{<}\ExtensionTok{img}\NormalTok{ src=}\StringTok{"cat.jpg"}\NormalTok{ alt=}\StringTok{"cat"}\NormalTok{ width=}\StringTok{"300"}\NormalTok{ height=}\StringTok{"200"}\OperatorTok{>}
        \OperatorTok{<}\ExtensionTok{img}\NormalTok{ src=}\StringTok{"cat.jpg"}\NormalTok{ alt=}\StringTok{"cat"}\NormalTok{ width=}\StringTok{"50%"} \OperatorTok{>}
    \OperatorTok{<}\NormalTok{/}\ExtensionTok{body}\OperatorTok{>}             
\OperatorTok{<}\NormalTok{/}\ExtensionTok{html}\OperatorTok{>}                 
\end{Highlighting}
\end{Shaded}

Tables display data in a table format that can be styled in various
ways. The \texttt{th} tag specifies the headings of each of the columns.
The \texttt{td} tag specifies the data in each of the columns.

\begin{Shaded}
\begin{Highlighting}[]
\OperatorTok{<}\NormalTok{!}\ExtensionTok{DOCTYPE}\NormalTok{ html}\OperatorTok{>}         
\OperatorTok{<}\ExtensionTok{html}\OperatorTok{>}                  
    \OperatorTok{<}\ExtensionTok{body}\OperatorTok{>}              
        \OperatorTok{<}\ExtensionTok{table}\OperatorTok{>}                         
            \OperatorTok{<}\FunctionTok{tr}\OperatorTok{>}                        
                \OperatorTok{<}\ExtensionTok{th}\OperatorTok{>}\NormalTok{First Name}\OperatorTok{<}\NormalTok{/th}\OperatorTok{>}     
                \OperatorTok{<}\ExtensionTok{th}\OperatorTok{>}\NormalTok{Last Name}\OperatorTok{<}\NormalTok{/th}\OperatorTok{>}      
                \OperatorTok{<}\ExtensionTok{th}\OperatorTok{>}\NormalTok{Years in Office}\OperatorTok{<}\NormalTok{/th}\OperatorTok{>}
            \OperatorTok{<}\NormalTok{/}\ExtensionTok{tr}\OperatorTok{>}                       
            \OperatorTok{<}\FunctionTok{tr}\OperatorTok{>}                        
                \OperatorTok{<}\ExtensionTok{td}\OperatorTok{>}\NormalTok{George}\OperatorTok{<}\NormalTok{/td}\OperatorTok{>}         
                \OperatorTok{<}\ExtensionTok{td}\OperatorTok{>}\NormalTok{Washington}\OperatorTok{<}\NormalTok{/td}\OperatorTok{>}     
                \OperatorTok{<}\ExtensionTok{td}\OperatorTok{>}\NormalTok{1789-}\OperatorTok{1797<}\NormalTok{/td}\OperatorTok{>}      
            \OperatorTok{<}\NormalTok{/}\ExtensionTok{tr}\OperatorTok{>}                       
            \OperatorTok{<}\FunctionTok{tr}\OperatorTok{>}                        
                \OperatorTok{<}\ExtensionTok{td}\OperatorTok{>}\NormalTok{John}\OperatorTok{<}\NormalTok{/td}\OperatorTok{>}           
                \OperatorTok{<}\ExtensionTok{td}\OperatorTok{>}\NormalTok{Adams}\OperatorTok{<}\NormalTok{/td}\OperatorTok{>}          
                \OperatorTok{<}\ExtensionTok{td}\OperatorTok{>}\NormalTok{1797-}\OperatorTok{1801<}\NormalTok{/td}\OperatorTok{>}      
            \OperatorTok{<}\NormalTok{/}\ExtensionTok{tr}\OperatorTok{>}                       
            \OperatorTok{<}\FunctionTok{tr}\OperatorTok{>}                        
                \OperatorTok{<}\ExtensionTok{td}\OperatorTok{>}\NormalTok{Thomas}\OperatorTok{<}\NormalTok{/td}\OperatorTok{>}         
                \OperatorTok{<}\ExtensionTok{td}\OperatorTok{>}\NormalTok{Jefferson}\OperatorTok{<}\NormalTok{/td}\OperatorTok{>}      
                \OperatorTok{<}\ExtensionTok{td}\OperatorTok{>}\NormalTok{1801-}\OperatorTok{1809<}\NormalTok{/td}\OperatorTok{>}      
            \OperatorTok{<}\NormalTok{/}\ExtensionTok{tr}\OperatorTok{>}                       
        \OperatorTok{<}\NormalTok{/}\ExtensionTok{table}\OperatorTok{>}                        
    \OperatorTok{<}\NormalTok{/}\ExtensionTok{body}\OperatorTok{>}             
\OperatorTok{<}\NormalTok{/}\ExtensionTok{html}\OperatorTok{>}                 
\end{Highlighting}
\end{Shaded}

Tables can be styled within the header of the html document. Both the
\texttt{th} and the \texttt{td} styles are defined together.
\texttt{border-collapse} combines the borders of cells together.

\begin{Shaded}
\begin{Highlighting}[]
\OperatorTok{<}\NormalTok{!}\ExtensionTok{DOCTYPE}\NormalTok{ html}\OperatorTok{>}         
\OperatorTok{<}\ExtensionTok{html}\OperatorTok{>}                  
    \OperatorTok{<}\FunctionTok{head}\OperatorTok{>}                                  
        \OperatorTok{<}\ExtensionTok{title}\OperatorTok{>}\NormalTok{Presidents}\OperatorTok{<}\NormalTok{/title}\OperatorTok{>}           
        \OperatorTok{<}\ExtensionTok{style}\OperatorTok{>}                             
            \ExtensionTok{table}\NormalTok{ \{                         }
                \ExtensionTok{border}\NormalTok{: 2px solid black}\KeywordTok{;}    
                \ExtensionTok{border-collapse}\NormalTok{: collapse}\KeywordTok{;}  
                \ExtensionTok{width}\NormalTok{: 50%}\KeywordTok{;}                 
\NormalTok{            \}                               }
                                            
            \ExtensionTok{th}\NormalTok{, td \{                        }
                \ExtensionTok{border}\NormalTok{: 1px solid black}\KeywordTok{;}    
                \ExtensionTok{padding}\NormalTok{: 5px}\KeywordTok{;}               
                \ExtensionTok{text-align}\NormalTok{: center}\KeywordTok{;}         
\NormalTok{            \}                               }
                                            
            \ExtensionTok{th}\NormalTok{ \{                            }
                \ExtensionTok{background-color}\NormalTok{: lightgray}\KeywordTok{;}
\NormalTok{            \}                               }
        \OperatorTok{<}\NormalTok{/}\ExtensionTok{style}\OperatorTok{>}                            
    \OperatorTok{<}\NormalTok{/}\ExtensionTok{head}\OperatorTok{>}                                 
    \OperatorTok{<}\ExtensionTok{body}\OperatorTok{>}              
        \OperatorTok{<}\ExtensionTok{table}\OperatorTok{>}                         
            \OperatorTok{<}\FunctionTok{tr}\OperatorTok{>}                        
                \OperatorTok{<}\ExtensionTok{th}\OperatorTok{>}\NormalTok{First Name}\OperatorTok{<}\NormalTok{/th}\OperatorTok{>}     
                \OperatorTok{<}\ExtensionTok{th}\OperatorTok{>}\NormalTok{Last Name}\OperatorTok{<}\NormalTok{/th}\OperatorTok{>}      
                \OperatorTok{<}\ExtensionTok{th}\OperatorTok{>}\NormalTok{Years in Office}\OperatorTok{<}\NormalTok{/th}\OperatorTok{>}
            \OperatorTok{<}\NormalTok{/}\ExtensionTok{tr}\OperatorTok{>}                       
            \OperatorTok{<}\FunctionTok{tr}\OperatorTok{>}                        
                \OperatorTok{<}\ExtensionTok{td}\OperatorTok{>}\NormalTok{George}\OperatorTok{<}\NormalTok{/td}\OperatorTok{>}         
                \OperatorTok{<}\ExtensionTok{td}\OperatorTok{>}\NormalTok{Washington}\OperatorTok{<}\NormalTok{/td}\OperatorTok{>}     
                \OperatorTok{<}\ExtensionTok{td}\OperatorTok{>}\NormalTok{1789-}\OperatorTok{1797<}\NormalTok{/td}\OperatorTok{>}      
            \OperatorTok{<}\NormalTok{/}\ExtensionTok{tr}\OperatorTok{>}                       
        \OperatorTok{<}\NormalTok{/}\ExtensionTok{table}\OperatorTok{>}                        
    \OperatorTok{<}\NormalTok{/}\ExtensionTok{body}\OperatorTok{>}             
\OperatorTok{<}\NormalTok{/}\ExtensionTok{html}\OperatorTok{>}                 
\end{Highlighting}
\end{Shaded}

Forms can be created and labeled as such. The \texttt{placeholder} text
is what is written within the form before anything is entered into it.
The \texttt{name} is similar to a variable name and can be used to refer
to the form and the data that is entered into it. The text within the
button is the text that will appear on the button in the page.

\begin{Shaded}
\begin{Highlighting}[]
\OperatorTok{<}\NormalTok{!}\ExtensionTok{DOCTYPE}\NormalTok{ html}\OperatorTok{>}         
\OperatorTok{<}\ExtensionTok{html}\OperatorTok{>}                  
    \OperatorTok{<}\ExtensionTok{body}\OperatorTok{>}              
        \OperatorTok{<}\ExtensionTok{form}\OperatorTok{>}                                                     
            \OperatorTok{<}\ExtensionTok{input}\NormalTok{ type=}\StringTok{"text"}\NormalTok{ placeholder=}\StringTok{"Full Name"}\NormalTok{ name=}\StringTok{"name"}\OperatorTok{>}
            \OperatorTok{<}\ExtensionTok{button}\OperatorTok{>}\NormalTok{Submit!}\OperatorTok{<}\NormalTok{/button}\OperatorTok{>}                               
        \OperatorTok{<}\NormalTok{/}\ExtensionTok{form}\OperatorTok{>}                                                    
    \OperatorTok{<}\NormalTok{/}\ExtensionTok{body}\OperatorTok{>}             
\OperatorTok{<}\NormalTok{/}\ExtensionTok{html}\OperatorTok{>}                 
\end{Highlighting}
\end{Shaded}

Text can be aligned and colord by specifying styles within the
respective tags of text.

\begin{Shaded}
\begin{Highlighting}[]
\OperatorTok{<}\NormalTok{!}\ExtensionTok{DOCTYPE}\NormalTok{ html}\OperatorTok{>}         
\OperatorTok{<}\ExtensionTok{html}\OperatorTok{>}                  
    \OperatorTok{<}\ExtensionTok{body}\OperatorTok{>}              
        \OperatorTok{<}\ExtensionTok{h1}\NormalTok{ style=}\StringTok{"color:red;text-align:center;"}\OperatorTok{>}\NormalTok{Welcome to My Web Page!}\OperatorTok{<}\NormalTok{/h1}\OperatorTok{>}
        \OperatorTok{<}\ExtensionTok{h1}\NormalTok{ style=}\StringTok{"color:#4290f5;text-align:center;"}\OperatorTok{>}\NormalTok{Second heading}\OperatorTok{<}\NormalTok{/h1}\OperatorTok{>}
    \OperatorTok{<}\NormalTok{/}\ExtensionTok{body}\OperatorTok{>}             
\OperatorTok{<}\NormalTok{/}\ExtensionTok{html}\OperatorTok{>}                 
\end{Highlighting}
\end{Shaded}

Style elements can be separated from the actual body of the webpage. In
this example every \texttt{h1} is styled within the style portion of the
header.

\begin{Shaded}
\begin{Highlighting}[]
\OperatorTok{<}\NormalTok{!}\ExtensionTok{DOCTYPE}\NormalTok{ html}\OperatorTok{>}         
\OperatorTok{<}\ExtensionTok{html}\OperatorTok{>}                  
    \OperatorTok{<}\FunctionTok{head}\OperatorTok{>}                         
        \OperatorTok{<}\ExtensionTok{title}\OperatorTok{>}\NormalTok{My Web Page!}\OperatorTok{<}\NormalTok{/title}\OperatorTok{>}
        \OperatorTok{<}\ExtensionTok{style}\OperatorTok{>}                    
            \ExtensionTok{h1}\NormalTok{ \{                   }
                \ExtensionTok{color}\NormalTok{: red}\KeywordTok{;}        
                \ExtensionTok{text-align}\NormalTok{: center}\KeywordTok{;}
\NormalTok{            \}                      }
        \OperatorTok{<}\NormalTok{/}\ExtensionTok{style}\OperatorTok{>}                   
    \OperatorTok{<}\NormalTok{/}\ExtensionTok{head}\OperatorTok{>}                        
    \OperatorTok{<}\ExtensionTok{body}\OperatorTok{>}              
        \OperatorTok{<}\ExtensionTok{h1}\OperatorTok{>}\NormalTok{Welcome to My Web Page!}\OperatorTok{<}\NormalTok{/h1}\OperatorTok{>}
    \OperatorTok{<}\NormalTok{/}\ExtensionTok{body}\OperatorTok{>}             
\OperatorTok{<}\NormalTok{/}\ExtensionTok{html}\OperatorTok{>}                 
\end{Highlighting}
\end{Shaded}

\section{CSS}\label{css}

CSS properties can be found
\href{https://developer.mozilla.org/en-US/docs/Web/CSS/Reference}{here}.

Instead of putting the css styles within the header of the html file,
they can be included in a separate css file and referenced. In this
example, the type of file being referenced is classified as a
\texttt{stylesheet} and the code is within \texttt{styles.css}.

\begin{Shaded}
\begin{Highlighting}[]
\OperatorTok{<}\NormalTok{!}\ExtensionTok{DOCTYPE}\NormalTok{ html}\OperatorTok{>}         
\OperatorTok{<}\ExtensionTok{html}\OperatorTok{>}                  
    \OperatorTok{<}\FunctionTok{head}\OperatorTok{>}                                       
        \OperatorTok{<}\FunctionTok{link}\NormalTok{ rel=}\StringTok{"stylesheet"}\NormalTok{ href=}\StringTok{"styles.css"}\OperatorTok{>}
    \OperatorTok{<}\NormalTok{/}\ExtensionTok{head}\OperatorTok{>}                                      
\OperatorTok{<}\NormalTok{/}\ExtensionTok{html}\OperatorTok{>}                 
\end{Highlighting}
\end{Shaded}

The code that goes within the css file is here, and it is simply the
same code that was put into the style headers in the above example.

\begin{Shaded}
\begin{Highlighting}[]
\ExtensionTok{h1}\NormalTok{ \{                   }
    \ExtensionTok{color}\NormalTok{: blue}\KeywordTok{;}       
    \ExtensionTok{text-align}\NormalTok{: center}\KeywordTok{;}
\NormalTok{\}                      }
\end{Highlighting}
\end{Shaded}

Divisions define sections of the code that can be separated so it can be
controlled in a particular manner. Font priorities are taken left to
right if some fonts are not found.

\begin{Shaded}
\begin{Highlighting}[]
\OperatorTok{<}\NormalTok{!}\ExtensionTok{DOCTYPE}\NormalTok{ html}\OperatorTok{>}                        
\OperatorTok{<}\ExtensionTok{html}\OperatorTok{>}                                 
    \OperatorTok{<}\FunctionTok{head}\OperatorTok{>}                             
        \OperatorTok{<}\ExtensionTok{title}\OperatorTok{>}\NormalTok{My Web Page!}\OperatorTok{<}\NormalTok{/title}\OperatorTok{>}    
        \OperatorTok{<}\ExtensionTok{style}\OperatorTok{>}                        
            \ExtensionTok{div}\NormalTok{ \{                      }
                \ExtensionTok{background-color}\NormalTok{: teal}\KeywordTok{;}
                \ExtensionTok{width}\NormalTok{: 500px}\KeywordTok{;}          
                \ExtensionTok{height}\NormalTok{: 400px}\KeywordTok{;}         
                \ExtensionTok{margin}\NormalTok{: 30px}\KeywordTok{;}
                \ExtensionTok{padding}\NormalTok{: 20px}\KeywordTok{;}
                \ExtensionTok{font-family}\NormalTok{: Arial, sans-serif}\KeywordTok{;}
                \ExtensionTok{font-size}\NormalTok{: 28px}\KeywordTok{;}               
                \ExtensionTok{font-weight}\NormalTok{: bold}\KeywordTok{;}             
                \ExtensionTok{border}\NormalTok{: 1px dotted black}\KeywordTok{;}
\NormalTok{            \}                          }
        \OperatorTok{<}\NormalTok{/}\ExtensionTok{style}\OperatorTok{>}                       
    \OperatorTok{<}\NormalTok{/}\ExtensionTok{head}\OperatorTok{>}                            
    \OperatorTok{<}\ExtensionTok{body}\OperatorTok{>}                             
        \OperatorTok{<}\ExtensionTok{div}\OperatorTok{>}                          
            \ExtensionTok{Hello}\NormalTok{, world!              }
        \OperatorTok{<}\NormalTok{/}\ExtensionTok{div}\OperatorTok{>}                         
    \OperatorTok{<}\NormalTok{/}\ExtensionTok{body}\OperatorTok{>}                            
\OperatorTok{<}\NormalTok{/}\ExtensionTok{html}\OperatorTok{>}                                
\end{Highlighting}
\end{Shaded}

Divisions and spans can be named and used to refer to different parts of
the html document specifically.

\begin{Shaded}
\begin{Highlighting}[]
\OperatorTok{<}\NormalTok{!}\ExtensionTok{DOCTYPE}\NormalTok{ html}\OperatorTok{>}                                                         
\OperatorTok{<}\ExtensionTok{html}\OperatorTok{>}                                                                  
    \OperatorTok{<}\FunctionTok{head}\OperatorTok{>}                                                              
        \OperatorTok{<}\ExtensionTok{title}\OperatorTok{>}\NormalTok{My Web Page!}\OperatorTok{<}\NormalTok{/title}\OperatorTok{>}                                     
        \OperatorTok{<}\ExtensionTok{style}\OperatorTok{>}                                                         
            \CommentTok{#top \{                                                      }
                \ExtensionTok{font-size}\NormalTok{: 36px}\KeywordTok{;}                                        
                \ExtensionTok{color}\NormalTok{: red}\KeywordTok{;}                                             
\NormalTok{            \}                                                           }
                                                                        
            \ExtensionTok{.name}\NormalTok{ \{                                                     }
                \ExtensionTok{font-weight}\NormalTok{: bold}\KeywordTok{;}                                      
                \ExtensionTok{color}\NormalTok{: blue}\KeywordTok{;}                                            
\NormalTok{            \}                                                           }
        \OperatorTok{<}\NormalTok{/}\ExtensionTok{style}\OperatorTok{>}                                                        
    \OperatorTok{<}\NormalTok{/}\ExtensionTok{head}\OperatorTok{>}                                                             
    \OperatorTok{<}\ExtensionTok{body}\OperatorTok{>}                                                              
        \OperatorTok{<}\ExtensionTok{div}\NormalTok{ id=}\StringTok{"top"}\OperatorTok{>}                                                  
            \ExtensionTok{This}\NormalTok{ is the }\OperatorTok{<}\NormalTok{span class=}\StringTok{"name"}\OperatorTok{>}\NormalTok{top}\OperatorTok{<}\NormalTok{/span}\OperatorTok{>}\NormalTok{ of my web page.   }
        \OperatorTok{<}\NormalTok{/}\ExtensionTok{div}\OperatorTok{>}                                                          
    \OperatorTok{<}\NormalTok{/}\ExtensionTok{body}\OperatorTok{>}                                                             
\OperatorTok{<}\NormalTok{/}\ExtensionTok{html}\OperatorTok{>}
\end{Highlighting}
\end{Shaded}

\chapter{Bootstrap}\label{bootstrap}

\section{Setup}\label{setup}

Create a folder that will contain the webfiles use npm to initialize a
\texttt{package.json} file. Follow through the prompts to add the
desired information. Set the entry point to be \texttt{index.html}. It
can also be helpful to add the node\_modules folder to
\texttt{.gitignore}.

\begin{Shaded}
\begin{Highlighting}[]
\ExtensionTok{npm}\NormalTok{ init}
\end{Highlighting}
\end{Shaded}

Then just initialize some basic \texttt{index.html} file for testing
purposes.

\begin{Shaded}
\begin{Highlighting}[]
\OperatorTok{<}\ExtensionTok{body}\OperatorTok{>}                        
    \OperatorTok{<}\ExtensionTok{h1}\OperatorTok{>}\NormalTok{This is a Header}\OperatorTok{<}\NormalTok{/h1}\OperatorTok{>} 
    \OperatorTok{<}\ExtensionTok{p}\OperatorTok{>}\NormalTok{This is a paragraph}\OperatorTok{<}\NormalTok{/p}\OperatorTok{>}
\OperatorTok{<}\NormalTok{/}\ExtensionTok{body}\OperatorTok{>}                       
\end{Highlighting}
\end{Shaded}

Install the lite server, which will serve up the content from the
folder. The \texttt{save-dev} flag will add the information to the json
file that the lite-server should be used to serve the content. This
should add lite-server under the devDependencies listing within
\texttt{package.json} and a node-modules folder.

\begin{Shaded}
\begin{Highlighting}[]
\ExtensionTok{npm}\NormalTok{ install lite-server --save-dev}
\end{Highlighting}
\end{Shaded}

Within the \texttt{package.json} file, add the start and lite listings
so it looks like the following.

\begin{Shaded}
\begin{Highlighting}[]
\StringTok{"scripts"}\NormalTok{: }\KeywordTok{\{}                                            
  \StringTok{"start"}\NormalTok{: }\StringTok{"npm run lite"}\NormalTok{,                              }
  \StringTok{"test"}\NormalTok{: }\StringTok{"echo }\DataTypeTok{\textbackslash{}"}\StringTok{Error: no test specified}\DataTypeTok{\textbackslash{}"}\StringTok{ && exit 1"}\NormalTok{,}
  \StringTok{"lite"}\NormalTok{: }\StringTok{"lite-server"}                                 
\KeywordTok{\}}\NormalTok{,                                                      }
\end{Highlighting}
\end{Shaded}

The lite-server can then be run using npm start.

\begin{Shaded}
\begin{Highlighting}[]
\ExtensionTok{npm}\NormalTok{ start}
\end{Highlighting}
\end{Shaded}

\chapter{Hackernews}\label{hackernews}

\section{Setup}\label{setup-1}

First make sure create react app is installed. The project here follows
\href{https://www.youtube.com/watch?v=oGB_VPrld0U\&list=PLTTC1K14KAxHj6AftnRUD28SQaoVauvl3}{this}
tutorial. There are lots of other good looking tutorials like
\href{https://www.freecodecamp.org/news/the-react-handbook-b71c27b0a795/}{The
React Handbook}, and others at
\href{https://gitconnected.com/learn/react}{gitconnected}.

\begin{Shaded}
\begin{Highlighting}[]
\ExtensionTok{npm}\NormalTok{ i -g create-react-app}
\end{Highlighting}
\end{Shaded}

Then create a new directory for the app.

\begin{Shaded}
\begin{Highlighting}[]
\ExtensionTok{create-react-app}\NormalTok{ hacker-news-clone}
\end{Highlighting}
\end{Shaded}

Change into the newly created directory and then create a file to handle
environmental variables.

\begin{Shaded}
\begin{Highlighting}[]
\BuiltInTok{cd}\NormalTok{ hacker-news-clone}
\FunctionTok{touch}\NormalTok{ .env}
\end{Highlighting}
\end{Shaded}

Within the \texttt{.env} file refer to the \texttt{src} folder. This
will allow dependencies to be more easily imported. Add the following to
the \texttt{.env} file.

\begin{Shaded}
\begin{Highlighting}[]
\VariableTok{NODE_PATH=}\NormalTok{src}
\end{Highlighting}
\end{Shaded}

Make a components directory within \texttt{src} to hold all of the
components for the project.

\begin{Shaded}
\begin{Highlighting}[]
\FunctionTok{mkdir}\NormalTok{ -p src/components/App}
\end{Highlighting}
\end{Shaded}

Make a services directory within \texttt{src} to add additional
functionality to the app and reference other site APIs.

\begin{Shaded}
\begin{Highlighting}[]
\FunctionTok{mkdir}\NormalTok{ src/services}
\end{Highlighting}
\end{Shaded}

Make a styles directory within \texttt{src} to add styles that can be
used across the app.

\begin{Shaded}
\begin{Highlighting}[]
\FunctionTok{mkdir}\NormalTok{ -p src/styles}
\end{Highlighting}
\end{Shaded}

Make a store directory within \texttt{src} to add styles that will add
Redux function.

\begin{Shaded}
\begin{Highlighting}[]
\FunctionTok{mkdir}\NormalTok{ -p src/store}
\end{Highlighting}
\end{Shaded}

Make a utils directory within \texttt{src} for shared functions across
the app.

\begin{Shaded}
\begin{Highlighting}[]
\FunctionTok{mkdir}\NormalTok{ -p src/utils}
\end{Highlighting}
\end{Shaded}

Now move \texttt{App.js} to components just to keep the components
bundled together. Rename \texttt{App.js} to index so that it can be
imported from the mycomponents app.

\begin{Shaded}
\begin{Highlighting}[]
\FunctionTok{mv}\NormalTok{ src/App*js src/components/App/}
\FunctionTok{mv}\NormalTok{ src/components/App/App.js src/components/App/index.js}
\FunctionTok{mv}\NormalTok{ src/logo.svg src/components/App/}
\end{Highlighting}
\end{Shaded}

Delete the css files because style components will be used instead.

\begin{Shaded}
\begin{Highlighting}[]
\FunctionTok{rm}\NormalTok{ src/*css}
\end{Highlighting}
\end{Shaded}

Remove the imports of the css files in
\texttt{src/components/App/index.js}.

\begin{Shaded}
\begin{Highlighting}[]
\ExtensionTok{import} \StringTok{'./App.css'}\KeywordTok{;}
\end{Highlighting}
\end{Shaded}

And remove the import within \texttt{src/index.js}.

\begin{Shaded}
\begin{Highlighting}[]
\ExtensionTok{import} \StringTok{'./index.css'}\KeywordTok{;}
\end{Highlighting}
\end{Shaded}

Now create some styles to be used throughout the app.

\begin{Shaded}
\begin{Highlighting}[]
\FunctionTok{mkdir}\NormalTok{ src/styles}
\FunctionTok{touch}\NormalTok{ src/styles/globals.js}
\FunctionTok{touch}\NormalTok{ src/styles/palette.js}
\end{Highlighting}
\end{Shaded}

The js files above contain routine code that can be copied from the
author's \href{https://github.com/treyhuffine/hn-clone}{github page}.
Alternatively, here is the code for \texttt{global.js}.

\begin{Shaded}
\begin{Highlighting}[]

\ExtensionTok{import}\NormalTok{ \{ injectGlobal \} }\ExtensionTok{from} \StringTok{'styled-components'}\KeywordTok{;}
\ExtensionTok{import}\NormalTok{ \{ colorsDark \} }\ExtensionTok{from} \StringTok{'./palette'}\KeywordTok{;}

\ExtensionTok{const}\NormalTok{ setGlobalStyles = () =}\OperatorTok{>}
  \ExtensionTok{injectGlobal}\KeywordTok{`}
    \ExtensionTok{*}\NormalTok{ \{}
      \ExtensionTok{box-sizing}\NormalTok{: border-box}\KeywordTok{;}
\NormalTok{    \}}
    \ExtensionTok{html}\NormalTok{, body \{}
      \ExtensionTok{font-family}\NormalTok{: Lato,Helvetica-Neue,Helvetica,Arial,sans-serif}\KeywordTok{;}
      \ExtensionTok{width}\NormalTok{: 100vw}\KeywordTok{;}
      \ExtensionTok{overflow-x}\NormalTok{: hidden}\KeywordTok{;}
      \ExtensionTok{margin}\NormalTok{: 0}\KeywordTok{;}
      \ExtensionTok{padding}\NormalTok{: 0}\KeywordTok{;}
      \ExtensionTok{min-height}\NormalTok{: 100vh}\KeywordTok{;}
      \ExtensionTok{background-color}\NormalTok{: }\VariableTok{$\{colorsDark}\ErrorTok{.background}\VariableTok{\}}\KeywordTok{;}
\NormalTok{    \}}
    \ExtensionTok{ul}\NormalTok{ \{}
      \ExtensionTok{list-style}\NormalTok{: none}\KeywordTok{;}
      \ExtensionTok{padding}\NormalTok{: 0}\KeywordTok{;}
\NormalTok{    \}}
    \ExtensionTok{a}\NormalTok{ \{}
      \ExtensionTok{text-decoration}\NormalTok{: none}\KeywordTok{;}
      \KeywordTok{&}\NormalTok{:}\ExtensionTok{visited}\NormalTok{ \{}
        \ExtensionTok{color}\NormalTok{: inherit}\KeywordTok{;}
\NormalTok{      \}}
\NormalTok{    \}}
  \KeywordTok{`;}

\BuiltInTok{export} \VariableTok{default} \VariableTok{setGlobalStyles}\NormalTok{;}
\end{Highlighting}
\end{Shaded}

And here is the code for \texttt{palette.js}.

\begin{Shaded}
\begin{Highlighting}[]
\BuiltInTok{export} \VariableTok{const} \VariableTok{colorsDark} \VariableTok{=}\NormalTok{ \{}
  \ExtensionTok{background}\NormalTok{: }\StringTok{'#272727'}\NormalTok{,}
  \ExtensionTok{backgroundSecondary}\NormalTok{: }\StringTok{'#393C3E'}\NormalTok{,}
  \ExtensionTok{text}\NormalTok{: }\StringTok{'#bfbebe'}\NormalTok{,}
  \ExtensionTok{textSecondary}\NormalTok{: }\StringTok{'#848886'}\NormalTok{,}
  \ExtensionTok{border}\NormalTok{: }\StringTok{'#272727'}\NormalTok{,}
\NormalTok{\};}

\BuiltInTok{export} \VariableTok{const} \VariableTok{colorsLight} \VariableTok{=}\NormalTok{ \{}
  \ExtensionTok{background}\NormalTok{: }\StringTok{'#EAEAEA'}\NormalTok{,}
  \ExtensionTok{backgroundSecondary}\NormalTok{: }\StringTok{'#F8F8F8'}\NormalTok{,}
  \ExtensionTok{text}\NormalTok{: }\StringTok{'#848886'}\NormalTok{,}
  \ExtensionTok{textSecondary}\NormalTok{: }\StringTok{'#aaaaaa'}\NormalTok{,}
  \ExtensionTok{border}\NormalTok{: }\StringTok{'#EAEAEA'}\NormalTok{,}
\NormalTok{\};}
\end{Highlighting}
\end{Shaded}

\bibliography{book.bib,packages.bib}


\end{document}
