\documentclass[]{book}
\usepackage{lmodern}
\usepackage{amssymb,amsmath}
\usepackage{ifxetex,ifluatex}
\usepackage{fixltx2e} % provides \textsubscript
\ifnum 0\ifxetex 1\fi\ifluatex 1\fi=0 % if pdftex
  \usepackage[T1]{fontenc}
  \usepackage[utf8]{inputenc}
\else % if luatex or xelatex
  \ifxetex
    \usepackage{mathspec}
  \else
    \usepackage{fontspec}
  \fi
  \defaultfontfeatures{Ligatures=TeX,Scale=MatchLowercase}
\fi
% use upquote if available, for straight quotes in verbatim environments
\IfFileExists{upquote.sty}{\usepackage{upquote}}{}
% use microtype if available
\IfFileExists{microtype.sty}{%
\usepackage{microtype}
\UseMicrotypeSet[protrusion]{basicmath} % disable protrusion for tt fonts
}{}
\usepackage[margin=1in]{geometry}
\usepackage{hyperref}
\hypersetup{unicode=true,
            pdftitle={Bootstrap and React for Web Development},
            pdfauthor={L A Liggett},
            pdfborder={0 0 0},
            breaklinks=true}
\urlstyle{same}  % don't use monospace font for urls
\usepackage{natbib}
\bibliographystyle{apalike}
\usepackage{color}
\usepackage{fancyvrb}
\newcommand{\VerbBar}{|}
\newcommand{\VERB}{\Verb[commandchars=\\\{\}]}
\DefineVerbatimEnvironment{Highlighting}{Verbatim}{commandchars=\\\{\}}
% Add ',fontsize=\small' for more characters per line
\usepackage{framed}
\definecolor{shadecolor}{RGB}{248,248,248}
\newenvironment{Shaded}{\begin{snugshade}}{\end{snugshade}}
\newcommand{\KeywordTok}[1]{\textcolor[rgb]{0.13,0.29,0.53}{\textbf{#1}}}
\newcommand{\DataTypeTok}[1]{\textcolor[rgb]{0.13,0.29,0.53}{#1}}
\newcommand{\DecValTok}[1]{\textcolor[rgb]{0.00,0.00,0.81}{#1}}
\newcommand{\BaseNTok}[1]{\textcolor[rgb]{0.00,0.00,0.81}{#1}}
\newcommand{\FloatTok}[1]{\textcolor[rgb]{0.00,0.00,0.81}{#1}}
\newcommand{\ConstantTok}[1]{\textcolor[rgb]{0.00,0.00,0.00}{#1}}
\newcommand{\CharTok}[1]{\textcolor[rgb]{0.31,0.60,0.02}{#1}}
\newcommand{\SpecialCharTok}[1]{\textcolor[rgb]{0.00,0.00,0.00}{#1}}
\newcommand{\StringTok}[1]{\textcolor[rgb]{0.31,0.60,0.02}{#1}}
\newcommand{\VerbatimStringTok}[1]{\textcolor[rgb]{0.31,0.60,0.02}{#1}}
\newcommand{\SpecialStringTok}[1]{\textcolor[rgb]{0.31,0.60,0.02}{#1}}
\newcommand{\ImportTok}[1]{#1}
\newcommand{\CommentTok}[1]{\textcolor[rgb]{0.56,0.35,0.01}{\textit{#1}}}
\newcommand{\DocumentationTok}[1]{\textcolor[rgb]{0.56,0.35,0.01}{\textbf{\textit{#1}}}}
\newcommand{\AnnotationTok}[1]{\textcolor[rgb]{0.56,0.35,0.01}{\textbf{\textit{#1}}}}
\newcommand{\CommentVarTok}[1]{\textcolor[rgb]{0.56,0.35,0.01}{\textbf{\textit{#1}}}}
\newcommand{\OtherTok}[1]{\textcolor[rgb]{0.56,0.35,0.01}{#1}}
\newcommand{\FunctionTok}[1]{\textcolor[rgb]{0.00,0.00,0.00}{#1}}
\newcommand{\VariableTok}[1]{\textcolor[rgb]{0.00,0.00,0.00}{#1}}
\newcommand{\ControlFlowTok}[1]{\textcolor[rgb]{0.13,0.29,0.53}{\textbf{#1}}}
\newcommand{\OperatorTok}[1]{\textcolor[rgb]{0.81,0.36,0.00}{\textbf{#1}}}
\newcommand{\BuiltInTok}[1]{#1}
\newcommand{\ExtensionTok}[1]{#1}
\newcommand{\PreprocessorTok}[1]{\textcolor[rgb]{0.56,0.35,0.01}{\textit{#1}}}
\newcommand{\AttributeTok}[1]{\textcolor[rgb]{0.77,0.63,0.00}{#1}}
\newcommand{\RegionMarkerTok}[1]{#1}
\newcommand{\InformationTok}[1]{\textcolor[rgb]{0.56,0.35,0.01}{\textbf{\textit{#1}}}}
\newcommand{\WarningTok}[1]{\textcolor[rgb]{0.56,0.35,0.01}{\textbf{\textit{#1}}}}
\newcommand{\AlertTok}[1]{\textcolor[rgb]{0.94,0.16,0.16}{#1}}
\newcommand{\ErrorTok}[1]{\textcolor[rgb]{0.64,0.00,0.00}{\textbf{#1}}}
\newcommand{\NormalTok}[1]{#1}
\usepackage{longtable,booktabs}
\usepackage{graphicx,grffile}
\makeatletter
\def\maxwidth{\ifdim\Gin@nat@width>\linewidth\linewidth\else\Gin@nat@width\fi}
\def\maxheight{\ifdim\Gin@nat@height>\textheight\textheight\else\Gin@nat@height\fi}
\makeatother
% Scale images if necessary, so that they will not overflow the page
% margins by default, and it is still possible to overwrite the defaults
% using explicit options in \includegraphics[width, height, ...]{}
\setkeys{Gin}{width=\maxwidth,height=\maxheight,keepaspectratio}
\IfFileExists{parskip.sty}{%
\usepackage{parskip}
}{% else
\setlength{\parindent}{0pt}
\setlength{\parskip}{6pt plus 2pt minus 1pt}
}
\setlength{\emergencystretch}{3em}  % prevent overfull lines
\providecommand{\tightlist}{%
  \setlength{\itemsep}{0pt}\setlength{\parskip}{0pt}}
\setcounter{secnumdepth}{5}
% Redefines (sub)paragraphs to behave more like sections
\ifx\paragraph\undefined\else
\let\oldparagraph\paragraph
\renewcommand{\paragraph}[1]{\oldparagraph{#1}\mbox{}}
\fi
\ifx\subparagraph\undefined\else
\let\oldsubparagraph\subparagraph
\renewcommand{\subparagraph}[1]{\oldsubparagraph{#1}\mbox{}}
\fi

%%% Use protect on footnotes to avoid problems with footnotes in titles
\let\rmarkdownfootnote\footnote%
\def\footnote{\protect\rmarkdownfootnote}

%%% Change title format to be more compact
\usepackage{titling}

% Create subtitle command for use in maketitle
\providecommand{\subtitle}[1]{
  \posttitle{
    \begin{center}\large#1\end{center}
    }
}

\setlength{\droptitle}{-2em}

  \title{Bootstrap and React for Web Development}
    \pretitle{\vspace{\droptitle}\centering\huge}
  \posttitle{\par}
    \author{L A Liggett}
    \preauthor{\centering\large\emph}
  \postauthor{\par}
      \predate{\centering\large\emph}
  \postdate{\par}
    \date{2019-07-16}

\usepackage{booktabs}
\usepackage{amsthm}
\makeatletter
\def\thm@space@setup{%
  \thm@preskip=8pt plus 2pt minus 4pt
  \thm@postskip=\thm@preskip
}
\makeatother

\begin{document}
\maketitle

{
\setcounter{tocdepth}{1}
\tableofcontents
}
\chapter{Introduction}\label{introduction}

\chapter{Bootstrap}\label{bootstrap}

\section{Setup}\label{setup}

Create a folder that will contain the webfiles use npm to initialize a
\texttt{package.json} file. Follow through the prompts to add the
desired information. Set the entry point to be \texttt{index.html}. It
can also be helpful to add the node\_modules folder to
\texttt{.gitignore}.

\begin{Shaded}
\begin{Highlighting}[]
\ExtensionTok{npm}\NormalTok{ init}
\end{Highlighting}
\end{Shaded}

Then just initialize some basic \texttt{index.html} file for testing
purposes.

\begin{Shaded}
\begin{Highlighting}[]
\OperatorTok{<}\ExtensionTok{body}\OperatorTok{>}                        
    \OperatorTok{<}\ExtensionTok{h1}\OperatorTok{>}\NormalTok{This is a Header}\OperatorTok{<}\NormalTok{/h1}\OperatorTok{>} 
    \OperatorTok{<}\ExtensionTok{p}\OperatorTok{>}\NormalTok{This is a paragraph}\OperatorTok{<}\NormalTok{/p}\OperatorTok{>}
\OperatorTok{<}\NormalTok{/}\ExtensionTok{body}\OperatorTok{>}                       
\end{Highlighting}
\end{Shaded}

Install the lite server, which will serve up the content from the
folder. The \texttt{save-dev} flag will add the information to the json
file that the lite-server should be used to serve the content. This
should add lite-server under the devDependencies listing within
\texttt{package.json} and a node-modules folder.

\begin{Shaded}
\begin{Highlighting}[]
\ExtensionTok{npm}\NormalTok{ install lite-server --save-dev}
\end{Highlighting}
\end{Shaded}

Within the \texttt{package.json} file, add the start and lite listings
so it looks like the following.

\begin{Shaded}
\begin{Highlighting}[]
\StringTok{"scripts"}\NormalTok{: }\KeywordTok{\{}                                            
  \StringTok{"start"}\NormalTok{: }\StringTok{"npm run lite"}\NormalTok{,                              }
  \StringTok{"test"}\NormalTok{: }\StringTok{"echo }\DataTypeTok{\textbackslash{}"}\StringTok{Error: no test specified}\DataTypeTok{\textbackslash{}"}\StringTok{ && exit 1"}\NormalTok{,}
  \StringTok{"lite"}\NormalTok{: }\StringTok{"lite-server"}                                 
\KeywordTok{\}}\NormalTok{,                                                      }
\end{Highlighting}
\end{Shaded}

The lite-server can then be run using npm start.

\begin{Shaded}
\begin{Highlighting}[]
\ExtensionTok{npm}\NormalTok{ start}
\end{Highlighting}
\end{Shaded}

\chapter{Hackernews}\label{hackernews}

\section{Setup}\label{setup-1}

First make sure create react app is installed. The project here follows
\href{https://www.youtube.com/watch?v=oGB_VPrld0U\&list=PLTTC1K14KAxHj6AftnRUD28SQaoVauvl3}{this}
tutorial. There are lots of other good looking tutorials like
\href{https://www.freecodecamp.org/news/the-react-handbook-b71c27b0a795/}{The
React Handbook}, and others at
\href{https://gitconnected.com/learn/react}{gitconnected}.

\begin{Shaded}
\begin{Highlighting}[]
\ExtensionTok{npm}\NormalTok{ i -g create-react-app}
\end{Highlighting}
\end{Shaded}

Then create a new directory for the app.

\begin{Shaded}
\begin{Highlighting}[]
\ExtensionTok{create-react-app}\NormalTok{ hacker-news-clone}
\end{Highlighting}
\end{Shaded}

Change into the newly created directory and then create a file to handle
environmental variables.

\begin{Shaded}
\begin{Highlighting}[]
\BuiltInTok{cd}\NormalTok{ hacker-news-clone}
\FunctionTok{touch}\NormalTok{ .env}
\end{Highlighting}
\end{Shaded}

Within the \texttt{.env} file refer to the \texttt{src} folder. This
will allow dependencies to be more easily imported. Add the following to
the \texttt{.env} file.

\begin{Shaded}
\begin{Highlighting}[]
\VariableTok{NODE_PATH=}\NormalTok{src}
\end{Highlighting}
\end{Shaded}

Make a components directory within \texttt{src} to hold all of the
components for the project.

\begin{Shaded}
\begin{Highlighting}[]
\FunctionTok{mkdir}\NormalTok{ -p src/components/App}
\end{Highlighting}
\end{Shaded}

Make a services directory within \texttt{src} to add additional
functionality to the app and reference other site APIs.

\begin{Shaded}
\begin{Highlighting}[]
\FunctionTok{mkdir}\NormalTok{ src/services}
\end{Highlighting}
\end{Shaded}

Make a styles directory within \texttt{src} to add styles that can be
used across the app.

\begin{Shaded}
\begin{Highlighting}[]
\FunctionTok{mkdir}\NormalTok{ -p src/styles}
\end{Highlighting}
\end{Shaded}

Make a store directory within \texttt{src} to add styles that will add
Redux function.

\begin{Shaded}
\begin{Highlighting}[]
\FunctionTok{mkdir}\NormalTok{ -p src/store}
\end{Highlighting}
\end{Shaded}

Make a utils directory within \texttt{src} for shared functions across
the app.

\begin{Shaded}
\begin{Highlighting}[]
\FunctionTok{mkdir}\NormalTok{ -p src/utils}
\end{Highlighting}
\end{Shaded}

Now move \texttt{App.js} to components just to keep the components
bundled together. Rename \texttt{App.js} to index so that it can be
imported from the mycomponents app.

\begin{Shaded}
\begin{Highlighting}[]
\FunctionTok{mv}\NormalTok{ src/App*js src/components/App/}
\FunctionTok{mv}\NormalTok{ src/components/App/App.js src/components/App/index.js}
\FunctionTok{mv}\NormalTok{ src/logo.svg src/components/App/}
\end{Highlighting}
\end{Shaded}

Delete the css files because style components will be used instead.

\begin{Shaded}
\begin{Highlighting}[]
\FunctionTok{rm}\NormalTok{ src/*css}
\end{Highlighting}
\end{Shaded}

Remove the imports of the css files in
\texttt{src/components/App/index.js}.

\begin{Shaded}
\begin{Highlighting}[]
\ExtensionTok{import} \StringTok{'./App.css'}\KeywordTok{;}
\end{Highlighting}
\end{Shaded}

And remove the import within \texttt{src/index.js}.

\begin{Shaded}
\begin{Highlighting}[]
\ExtensionTok{import} \StringTok{'./index.css'}\KeywordTok{;}
\end{Highlighting}
\end{Shaded}

\bibliography{book.bib,packages.bib}


\end{document}
